
% Default to the notebook output style

    


% Inherit from the specified cell style.




    
\documentclass[11pt]{article}

    
    
    \usepackage[T1]{fontenc}
    % Nicer default font (+ math font) than Computer Modern for most use cases
    \usepackage{mathpazo}

    % Basic figure setup, for now with no caption control since it's done
    % automatically by Pandoc (which extracts ![](path) syntax from Markdown).
    \usepackage{graphicx}
    % We will generate all images so they have a width \maxwidth. This means
    % that they will get their normal width if they fit onto the page, but
    % are scaled down if they would overflow the margins.
    \makeatletter
    \def\maxwidth{\ifdim\Gin@nat@width>\linewidth\linewidth
    \else\Gin@nat@width\fi}
    \makeatother
    \let\Oldincludegraphics\includegraphics
    % Set max figure width to be 80% of text width, for now hardcoded.
    \renewcommand{\includegraphics}[1]{\Oldincludegraphics[width=.8\maxwidth]{#1}}
    % Ensure that by default, figures have no caption (until we provide a
    % proper Figure object with a Caption API and a way to capture that
    % in the conversion process - todo).
    \usepackage{caption}
    \DeclareCaptionLabelFormat{nolabel}{}
    \captionsetup{labelformat=nolabel}

    \usepackage{adjustbox} % Used to constrain images to a maximum size 
    \usepackage{xcolor} % Allow colors to be defined
    \usepackage{enumerate} % Needed for markdown enumerations to work
    \usepackage{geometry} % Used to adjust the document margins
    \usepackage{amsmath} % Equations
    \usepackage{amssymb} % Equations
    \usepackage{textcomp} % defines textquotesingle
    % Hack from http://tex.stackexchange.com/a/47451/13684:
    \AtBeginDocument{%
        \def\PYZsq{\textquotesingle}% Upright quotes in Pygmentized code
    }
    \usepackage{upquote} % Upright quotes for verbatim code
    \usepackage{eurosym} % defines \euro
    \usepackage[mathletters]{ucs} % Extended unicode (utf-8) support
    \usepackage[utf8x]{inputenc} % Allow utf-8 characters in the tex document
    \usepackage{fancyvrb} % verbatim replacement that allows latex
    \usepackage{grffile} % extends the file name processing of package graphics 
                         % to support a larger range 
    % The hyperref package gives us a pdf with properly built
    % internal navigation ('pdf bookmarks' for the table of contents,
    % internal cross-reference links, web links for URLs, etc.)
    \usepackage{hyperref}
    \usepackage{longtable} % longtable support required by pandoc >1.10
    \usepackage{booktabs}  % table support for pandoc > 1.12.2
    \usepackage[inline]{enumitem} % IRkernel/repr support (it uses the enumerate* environment)
    \usepackage[normalem]{ulem} % ulem is needed to support strikethroughs (\sout)
                                % normalem makes italics be italics, not underlines
    

    
    
    % Colors for the hyperref package
    \definecolor{urlcolor}{rgb}{0,.145,.698}
    \definecolor{linkcolor}{rgb}{.71,0.21,0.01}
    \definecolor{citecolor}{rgb}{.12,.54,.11}

    % ANSI colors
    \definecolor{ansi-black}{HTML}{3E424D}
    \definecolor{ansi-black-intense}{HTML}{282C36}
    \definecolor{ansi-red}{HTML}{E75C58}
    \definecolor{ansi-red-intense}{HTML}{B22B31}
    \definecolor{ansi-green}{HTML}{00A250}
    \definecolor{ansi-green-intense}{HTML}{007427}
    \definecolor{ansi-yellow}{HTML}{DDB62B}
    \definecolor{ansi-yellow-intense}{HTML}{B27D12}
    \definecolor{ansi-blue}{HTML}{208FFB}
    \definecolor{ansi-blue-intense}{HTML}{0065CA}
    \definecolor{ansi-magenta}{HTML}{D160C4}
    \definecolor{ansi-magenta-intense}{HTML}{A03196}
    \definecolor{ansi-cyan}{HTML}{60C6C8}
    \definecolor{ansi-cyan-intense}{HTML}{258F8F}
    \definecolor{ansi-white}{HTML}{C5C1B4}
    \definecolor{ansi-white-intense}{HTML}{A1A6B2}

    % commands and environments needed by pandoc snippets
    % extracted from the output of `pandoc -s`
    \providecommand{\tightlist}{%
      \setlength{\itemsep}{0pt}\setlength{\parskip}{0pt}}
    \DefineVerbatimEnvironment{Highlighting}{Verbatim}{commandchars=\\\{\}}
    % Add ',fontsize=\small' for more characters per line
    \newenvironment{Shaded}{}{}
    \newcommand{\KeywordTok}[1]{\textcolor[rgb]{0.00,0.44,0.13}{\textbf{{#1}}}}
    \newcommand{\DataTypeTok}[1]{\textcolor[rgb]{0.56,0.13,0.00}{{#1}}}
    \newcommand{\DecValTok}[1]{\textcolor[rgb]{0.25,0.63,0.44}{{#1}}}
    \newcommand{\BaseNTok}[1]{\textcolor[rgb]{0.25,0.63,0.44}{{#1}}}
    \newcommand{\FloatTok}[1]{\textcolor[rgb]{0.25,0.63,0.44}{{#1}}}
    \newcommand{\CharTok}[1]{\textcolor[rgb]{0.25,0.44,0.63}{{#1}}}
    \newcommand{\StringTok}[1]{\textcolor[rgb]{0.25,0.44,0.63}{{#1}}}
    \newcommand{\CommentTok}[1]{\textcolor[rgb]{0.38,0.63,0.69}{\textit{{#1}}}}
    \newcommand{\OtherTok}[1]{\textcolor[rgb]{0.00,0.44,0.13}{{#1}}}
    \newcommand{\AlertTok}[1]{\textcolor[rgb]{1.00,0.00,0.00}{\textbf{{#1}}}}
    \newcommand{\FunctionTok}[1]{\textcolor[rgb]{0.02,0.16,0.49}{{#1}}}
    \newcommand{\RegionMarkerTok}[1]{{#1}}
    \newcommand{\ErrorTok}[1]{\textcolor[rgb]{1.00,0.00,0.00}{\textbf{{#1}}}}
    \newcommand{\NormalTok}[1]{{#1}}
    
    % Additional commands for more recent versions of Pandoc
    \newcommand{\ConstantTok}[1]{\textcolor[rgb]{0.53,0.00,0.00}{{#1}}}
    \newcommand{\SpecialCharTok}[1]{\textcolor[rgb]{0.25,0.44,0.63}{{#1}}}
    \newcommand{\VerbatimStringTok}[1]{\textcolor[rgb]{0.25,0.44,0.63}{{#1}}}
    \newcommand{\SpecialStringTok}[1]{\textcolor[rgb]{0.73,0.40,0.53}{{#1}}}
    \newcommand{\ImportTok}[1]{{#1}}
    \newcommand{\DocumentationTok}[1]{\textcolor[rgb]{0.73,0.13,0.13}{\textit{{#1}}}}
    \newcommand{\AnnotationTok}[1]{\textcolor[rgb]{0.38,0.63,0.69}{\textbf{\textit{{#1}}}}}
    \newcommand{\CommentVarTok}[1]{\textcolor[rgb]{0.38,0.63,0.69}{\textbf{\textit{{#1}}}}}
    \newcommand{\VariableTok}[1]{\textcolor[rgb]{0.10,0.09,0.49}{{#1}}}
    \newcommand{\ControlFlowTok}[1]{\textcolor[rgb]{0.00,0.44,0.13}{\textbf{{#1}}}}
    \newcommand{\OperatorTok}[1]{\textcolor[rgb]{0.40,0.40,0.40}{{#1}}}
    \newcommand{\BuiltInTok}[1]{{#1}}
    \newcommand{\ExtensionTok}[1]{{#1}}
    \newcommand{\PreprocessorTok}[1]{\textcolor[rgb]{0.74,0.48,0.00}{{#1}}}
    \newcommand{\AttributeTok}[1]{\textcolor[rgb]{0.49,0.56,0.16}{{#1}}}
    \newcommand{\InformationTok}[1]{\textcolor[rgb]{0.38,0.63,0.69}{\textbf{\textit{{#1}}}}}
    \newcommand{\WarningTok}[1]{\textcolor[rgb]{0.38,0.63,0.69}{\textbf{\textit{{#1}}}}}
    
    
    % Define a nice break command that doesn't care if a line doesn't already
    % exist.
    \def\br{\hspace*{\fill} \\* }
    % Math Jax compatability definitions
    \def\gt{>}
    \def\lt{<}
    % Document parameters
    \title{CometLanding Hashtag Tweet Analysis}
    
    
    

    % Pygments definitions
    
\makeatletter
\def\PY@reset{\let\PY@it=\relax \let\PY@bf=\relax%
    \let\PY@ul=\relax \let\PY@tc=\relax%
    \let\PY@bc=\relax \let\PY@ff=\relax}
\def\PY@tok#1{\csname PY@tok@#1\endcsname}
\def\PY@toks#1+{\ifx\relax#1\empty\else%
    \PY@tok{#1}\expandafter\PY@toks\fi}
\def\PY@do#1{\PY@bc{\PY@tc{\PY@ul{%
    \PY@it{\PY@bf{\PY@ff{#1}}}}}}}
\def\PY#1#2{\PY@reset\PY@toks#1+\relax+\PY@do{#2}}

\expandafter\def\csname PY@tok@w\endcsname{\def\PY@tc##1{\textcolor[rgb]{0.73,0.73,0.73}{##1}}}
\expandafter\def\csname PY@tok@c\endcsname{\let\PY@it=\textit\def\PY@tc##1{\textcolor[rgb]{0.25,0.50,0.50}{##1}}}
\expandafter\def\csname PY@tok@cp\endcsname{\def\PY@tc##1{\textcolor[rgb]{0.74,0.48,0.00}{##1}}}
\expandafter\def\csname PY@tok@k\endcsname{\let\PY@bf=\textbf\def\PY@tc##1{\textcolor[rgb]{0.00,0.50,0.00}{##1}}}
\expandafter\def\csname PY@tok@kp\endcsname{\def\PY@tc##1{\textcolor[rgb]{0.00,0.50,0.00}{##1}}}
\expandafter\def\csname PY@tok@kt\endcsname{\def\PY@tc##1{\textcolor[rgb]{0.69,0.00,0.25}{##1}}}
\expandafter\def\csname PY@tok@o\endcsname{\def\PY@tc##1{\textcolor[rgb]{0.40,0.40,0.40}{##1}}}
\expandafter\def\csname PY@tok@ow\endcsname{\let\PY@bf=\textbf\def\PY@tc##1{\textcolor[rgb]{0.67,0.13,1.00}{##1}}}
\expandafter\def\csname PY@tok@nb\endcsname{\def\PY@tc##1{\textcolor[rgb]{0.00,0.50,0.00}{##1}}}
\expandafter\def\csname PY@tok@nf\endcsname{\def\PY@tc##1{\textcolor[rgb]{0.00,0.00,1.00}{##1}}}
\expandafter\def\csname PY@tok@nc\endcsname{\let\PY@bf=\textbf\def\PY@tc##1{\textcolor[rgb]{0.00,0.00,1.00}{##1}}}
\expandafter\def\csname PY@tok@nn\endcsname{\let\PY@bf=\textbf\def\PY@tc##1{\textcolor[rgb]{0.00,0.00,1.00}{##1}}}
\expandafter\def\csname PY@tok@ne\endcsname{\let\PY@bf=\textbf\def\PY@tc##1{\textcolor[rgb]{0.82,0.25,0.23}{##1}}}
\expandafter\def\csname PY@tok@nv\endcsname{\def\PY@tc##1{\textcolor[rgb]{0.10,0.09,0.49}{##1}}}
\expandafter\def\csname PY@tok@no\endcsname{\def\PY@tc##1{\textcolor[rgb]{0.53,0.00,0.00}{##1}}}
\expandafter\def\csname PY@tok@nl\endcsname{\def\PY@tc##1{\textcolor[rgb]{0.63,0.63,0.00}{##1}}}
\expandafter\def\csname PY@tok@ni\endcsname{\let\PY@bf=\textbf\def\PY@tc##1{\textcolor[rgb]{0.60,0.60,0.60}{##1}}}
\expandafter\def\csname PY@tok@na\endcsname{\def\PY@tc##1{\textcolor[rgb]{0.49,0.56,0.16}{##1}}}
\expandafter\def\csname PY@tok@nt\endcsname{\let\PY@bf=\textbf\def\PY@tc##1{\textcolor[rgb]{0.00,0.50,0.00}{##1}}}
\expandafter\def\csname PY@tok@nd\endcsname{\def\PY@tc##1{\textcolor[rgb]{0.67,0.13,1.00}{##1}}}
\expandafter\def\csname PY@tok@s\endcsname{\def\PY@tc##1{\textcolor[rgb]{0.73,0.13,0.13}{##1}}}
\expandafter\def\csname PY@tok@sd\endcsname{\let\PY@it=\textit\def\PY@tc##1{\textcolor[rgb]{0.73,0.13,0.13}{##1}}}
\expandafter\def\csname PY@tok@si\endcsname{\let\PY@bf=\textbf\def\PY@tc##1{\textcolor[rgb]{0.73,0.40,0.53}{##1}}}
\expandafter\def\csname PY@tok@se\endcsname{\let\PY@bf=\textbf\def\PY@tc##1{\textcolor[rgb]{0.73,0.40,0.13}{##1}}}
\expandafter\def\csname PY@tok@sr\endcsname{\def\PY@tc##1{\textcolor[rgb]{0.73,0.40,0.53}{##1}}}
\expandafter\def\csname PY@tok@ss\endcsname{\def\PY@tc##1{\textcolor[rgb]{0.10,0.09,0.49}{##1}}}
\expandafter\def\csname PY@tok@sx\endcsname{\def\PY@tc##1{\textcolor[rgb]{0.00,0.50,0.00}{##1}}}
\expandafter\def\csname PY@tok@m\endcsname{\def\PY@tc##1{\textcolor[rgb]{0.40,0.40,0.40}{##1}}}
\expandafter\def\csname PY@tok@gh\endcsname{\let\PY@bf=\textbf\def\PY@tc##1{\textcolor[rgb]{0.00,0.00,0.50}{##1}}}
\expandafter\def\csname PY@tok@gu\endcsname{\let\PY@bf=\textbf\def\PY@tc##1{\textcolor[rgb]{0.50,0.00,0.50}{##1}}}
\expandafter\def\csname PY@tok@gd\endcsname{\def\PY@tc##1{\textcolor[rgb]{0.63,0.00,0.00}{##1}}}
\expandafter\def\csname PY@tok@gi\endcsname{\def\PY@tc##1{\textcolor[rgb]{0.00,0.63,0.00}{##1}}}
\expandafter\def\csname PY@tok@gr\endcsname{\def\PY@tc##1{\textcolor[rgb]{1.00,0.00,0.00}{##1}}}
\expandafter\def\csname PY@tok@ge\endcsname{\let\PY@it=\textit}
\expandafter\def\csname PY@tok@gs\endcsname{\let\PY@bf=\textbf}
\expandafter\def\csname PY@tok@gp\endcsname{\let\PY@bf=\textbf\def\PY@tc##1{\textcolor[rgb]{0.00,0.00,0.50}{##1}}}
\expandafter\def\csname PY@tok@go\endcsname{\def\PY@tc##1{\textcolor[rgb]{0.53,0.53,0.53}{##1}}}
\expandafter\def\csname PY@tok@gt\endcsname{\def\PY@tc##1{\textcolor[rgb]{0.00,0.27,0.87}{##1}}}
\expandafter\def\csname PY@tok@err\endcsname{\def\PY@bc##1{\setlength{\fboxsep}{0pt}\fcolorbox[rgb]{1.00,0.00,0.00}{1,1,1}{\strut ##1}}}
\expandafter\def\csname PY@tok@kc\endcsname{\let\PY@bf=\textbf\def\PY@tc##1{\textcolor[rgb]{0.00,0.50,0.00}{##1}}}
\expandafter\def\csname PY@tok@kd\endcsname{\let\PY@bf=\textbf\def\PY@tc##1{\textcolor[rgb]{0.00,0.50,0.00}{##1}}}
\expandafter\def\csname PY@tok@kn\endcsname{\let\PY@bf=\textbf\def\PY@tc##1{\textcolor[rgb]{0.00,0.50,0.00}{##1}}}
\expandafter\def\csname PY@tok@kr\endcsname{\let\PY@bf=\textbf\def\PY@tc##1{\textcolor[rgb]{0.00,0.50,0.00}{##1}}}
\expandafter\def\csname PY@tok@bp\endcsname{\def\PY@tc##1{\textcolor[rgb]{0.00,0.50,0.00}{##1}}}
\expandafter\def\csname PY@tok@fm\endcsname{\def\PY@tc##1{\textcolor[rgb]{0.00,0.00,1.00}{##1}}}
\expandafter\def\csname PY@tok@vc\endcsname{\def\PY@tc##1{\textcolor[rgb]{0.10,0.09,0.49}{##1}}}
\expandafter\def\csname PY@tok@vg\endcsname{\def\PY@tc##1{\textcolor[rgb]{0.10,0.09,0.49}{##1}}}
\expandafter\def\csname PY@tok@vi\endcsname{\def\PY@tc##1{\textcolor[rgb]{0.10,0.09,0.49}{##1}}}
\expandafter\def\csname PY@tok@vm\endcsname{\def\PY@tc##1{\textcolor[rgb]{0.10,0.09,0.49}{##1}}}
\expandafter\def\csname PY@tok@sa\endcsname{\def\PY@tc##1{\textcolor[rgb]{0.73,0.13,0.13}{##1}}}
\expandafter\def\csname PY@tok@sb\endcsname{\def\PY@tc##1{\textcolor[rgb]{0.73,0.13,0.13}{##1}}}
\expandafter\def\csname PY@tok@sc\endcsname{\def\PY@tc##1{\textcolor[rgb]{0.73,0.13,0.13}{##1}}}
\expandafter\def\csname PY@tok@dl\endcsname{\def\PY@tc##1{\textcolor[rgb]{0.73,0.13,0.13}{##1}}}
\expandafter\def\csname PY@tok@s2\endcsname{\def\PY@tc##1{\textcolor[rgb]{0.73,0.13,0.13}{##1}}}
\expandafter\def\csname PY@tok@sh\endcsname{\def\PY@tc##1{\textcolor[rgb]{0.73,0.13,0.13}{##1}}}
\expandafter\def\csname PY@tok@s1\endcsname{\def\PY@tc##1{\textcolor[rgb]{0.73,0.13,0.13}{##1}}}
\expandafter\def\csname PY@tok@mb\endcsname{\def\PY@tc##1{\textcolor[rgb]{0.40,0.40,0.40}{##1}}}
\expandafter\def\csname PY@tok@mf\endcsname{\def\PY@tc##1{\textcolor[rgb]{0.40,0.40,0.40}{##1}}}
\expandafter\def\csname PY@tok@mh\endcsname{\def\PY@tc##1{\textcolor[rgb]{0.40,0.40,0.40}{##1}}}
\expandafter\def\csname PY@tok@mi\endcsname{\def\PY@tc##1{\textcolor[rgb]{0.40,0.40,0.40}{##1}}}
\expandafter\def\csname PY@tok@il\endcsname{\def\PY@tc##1{\textcolor[rgb]{0.40,0.40,0.40}{##1}}}
\expandafter\def\csname PY@tok@mo\endcsname{\def\PY@tc##1{\textcolor[rgb]{0.40,0.40,0.40}{##1}}}
\expandafter\def\csname PY@tok@ch\endcsname{\let\PY@it=\textit\def\PY@tc##1{\textcolor[rgb]{0.25,0.50,0.50}{##1}}}
\expandafter\def\csname PY@tok@cm\endcsname{\let\PY@it=\textit\def\PY@tc##1{\textcolor[rgb]{0.25,0.50,0.50}{##1}}}
\expandafter\def\csname PY@tok@cpf\endcsname{\let\PY@it=\textit\def\PY@tc##1{\textcolor[rgb]{0.25,0.50,0.50}{##1}}}
\expandafter\def\csname PY@tok@c1\endcsname{\let\PY@it=\textit\def\PY@tc##1{\textcolor[rgb]{0.25,0.50,0.50}{##1}}}
\expandafter\def\csname PY@tok@cs\endcsname{\let\PY@it=\textit\def\PY@tc##1{\textcolor[rgb]{0.25,0.50,0.50}{##1}}}

\def\PYZbs{\char`\\}
\def\PYZus{\char`\_}
\def\PYZob{\char`\{}
\def\PYZcb{\char`\}}
\def\PYZca{\char`\^}
\def\PYZam{\char`\&}
\def\PYZlt{\char`\<}
\def\PYZgt{\char`\>}
\def\PYZsh{\char`\#}
\def\PYZpc{\char`\%}
\def\PYZdl{\char`\$}
\def\PYZhy{\char`\-}
\def\PYZsq{\char`\'}
\def\PYZdq{\char`\"}
\def\PYZti{\char`\~}
% for compatibility with earlier versions
\def\PYZat{@}
\def\PYZlb{[}
\def\PYZrb{]}
\makeatother


    % Exact colors from NB
    \definecolor{incolor}{rgb}{0.0, 0.0, 0.5}
    \definecolor{outcolor}{rgb}{0.545, 0.0, 0.0}



    
    % Prevent overflowing lines due to hard-to-break entities
    \sloppy 
    % Setup hyperref package
    \hypersetup{
      breaklinks=true,  % so long urls are correctly broken across lines
      colorlinks=true,
      urlcolor=urlcolor,
      linkcolor=linkcolor,
      citecolor=citecolor,
      }
    % Slightly bigger margins than the latex defaults
    
    \geometry{verbose,tmargin=1in,bmargin=1in,lmargin=1in,rmargin=1in}
    
    

    \begin{document}
    
    
    \maketitle
    
    

    
    Analysis of Tweets Containing the CometLanding Hashtag

    \begin{Verbatim}[commandchars=\\\{\}]
{\color{incolor}In [{\color{incolor}1}]:} \PY{o}{\PYZpc{}}\PY{k}{matplotlib} inline
\end{Verbatim}


    Preprocessing input CSV data

    \begin{Verbatim}[commandchars=\\\{\}]
{\color{incolor}In [{\color{incolor}2}]:} \PY{k+kn}{from} \PY{n+nn}{parsing} \PY{k}{import} \PY{o}{*}
        \PY{n}{cr} \PY{o}{=} \PY{n}{CensusReader}\PY{p}{(}\PY{l+s+s2}{\PYZdq{}}\PY{l+s+s2}{data/CometLanding.csv}\PY{l+s+s2}{\PYZdq{}}\PY{p}{)}
\end{Verbatim}


    \begin{Verbatim}[commandchars=\\\{\}]
File contains 51 invalid rows!
Refining the data given and storing it a new file{\ldots}
Refined data can be found in 'data/CometLanding\_refined.csv'.

    \end{Verbatim}

    CensusReader class \textbf{init}() method

The \textbf{init}() method in the CensusReader class calls read\_csv()
from pandas on the file CometLanding.csv. This creates a DataFrame which
is then parsed with validateFile(); invalid tweets are removed in this
process.

    \begin{Verbatim}[commandchars=\\\{\}]
{\color{incolor}In [{\color{incolor}3}]:} \PY{k}{def} \PY{n+nf}{\PYZus{}\PYZus{}init\PYZus{}\PYZus{}}\PY{p}{(}\PY{n+nb+bp}{self}\PY{p}{,} \PY{n}{fileName}\PY{p}{)}\PY{p}{:}
            \PY{n+nb+bp}{self}\PY{o}{.}\PY{n}{fileName} \PY{o}{=} \PY{n}{fileName}
            \PY{n+nb+bp}{self}\PY{o}{.}\PY{n}{data} \PY{o}{=} \PY{n}{pd}\PY{o}{.}\PY{n}{read\PYZus{}csv}\PY{p}{(}\PY{n}{fileName}\PY{p}{)}
            
            \PY{n+nb+bp}{self}\PY{o}{.}\PY{n}{data}\PY{p}{[}\PY{l+s+s2}{\PYZdq{}}\PY{l+s+s2}{valid}\PY{l+s+s2}{\PYZdq{}}\PY{p}{]} \PY{o}{=} \PY{k+kc}{True}
            \PY{n+nb+bp}{self}\PY{o}{.}\PY{n}{validateFile}\PY{p}{(}\PY{p}{)}
\end{Verbatim}


    validateFile() method

The validateFile() method finds all duplicte entries in the DataFrame
(by their string tweet ID). It then removes all the duplicate entries
from the DataFrame, before printing the number of duplicate tweets
(invalid rows) there were. Finally, it outputs the remaining (valid,
non-duplicate) entries/tweets in the DataFrame to a new CSV file. It
outputs it to the filename the data was read in from, with \_refined
added before the .csv extension; the data is converted from a DataFrame
into CSV format using the to\_csv() method.

    \begin{Verbatim}[commandchars=\\\{\}]
{\color{incolor}In [{\color{incolor}4}]:} \PY{k}{def} \PY{n+nf}{validateFile}\PY{p}{(}\PY{n+nb+bp}{self}\PY{p}{)}\PY{p}{:}
                \PY{c+c1}{\PYZsh{}check duplication}
                \PY{n+nb+bp}{self}\PY{o}{.}\PY{n}{data}\PY{p}{[}\PY{l+s+s2}{\PYZdq{}}\PY{l+s+s2}{valid}\PY{l+s+s2}{\PYZdq{}}\PY{p}{]} \PY{o}{=} \PY{n+nb+bp}{self}\PY{o}{.}\PY{n}{data}\PY{p}{[}\PY{l+s+s2}{\PYZdq{}}\PY{l+s+s2}{valid}\PY{l+s+s2}{\PYZdq{}}\PY{p}{]} \PY{o}{\PYZam{}} \PY{o}{\PYZti{}}\PY{p}{(}\PY{n+nb+bp}{self}\PY{o}{.}\PY{n}{data}\PY{o}{.}\PY{n}{duplicated}\PY{p}{(}\PY{l+s+s2}{\PYZdq{}}\PY{l+s+s2}{id\PYZus{}str}\PY{l+s+s2}{\PYZdq{}}\PY{p}{)}\PY{p}{)}
        
                \PY{n}{invalidData} \PY{o}{=} \PY{n+nb+bp}{self}\PY{o}{.}\PY{n}{data}\PY{o}{.}\PY{n}{query}\PY{p}{(}\PY{l+s+s2}{\PYZdq{}}\PY{l+s+s2}{valid == False}\PY{l+s+s2}{\PYZdq{}}\PY{p}{)}
                \PY{n+nb+bp}{self}\PY{o}{.}\PY{n}{data} \PY{o}{=} \PY{n+nb+bp}{self}\PY{o}{.}\PY{n}{data}\PY{o}{.}\PY{n}{query}\PY{p}{(}\PY{l+s+s2}{\PYZdq{}}\PY{l+s+s2}{valid == True}\PY{l+s+s2}{\PYZdq{}}\PY{p}{)}
        
                \PY{c+c1}{\PYZsh{} Drops the \PYZsq{}valid\PYZsq{} column since it is no longer needed.}
                \PY{n+nb+bp}{self}\PY{o}{.}\PY{n}{data}\PY{o}{.}\PY{n}{drop}\PY{p}{(}\PY{l+s+s1}{\PYZsq{}}\PY{l+s+s1}{valid}\PY{l+s+s1}{\PYZsq{}}\PY{p}{,} \PY{n}{axis}\PY{o}{=}\PY{l+m+mi}{1}\PY{p}{,} \PY{n}{inplace}\PY{o}{=}\PY{k+kc}{True}\PY{p}{)}
        
                \PY{n}{invalidRows} \PY{o}{=} \PY{n+nb}{len}\PY{p}{(}\PY{n}{invalidData}\PY{p}{)}
        
                \PY{k}{if} \PY{p}{(}\PY{n}{invalidRows} \PY{o}{\PYZgt{}} \PY{l+m+mi}{0}\PY{p}{)}\PY{p}{:}
                    \PY{k}{if} \PY{n}{invalidRows} \PY{o}{==} \PY{l+m+mi}{1}\PY{p}{:}
                        \PY{n+nb}{print}\PY{p}{(}\PY{l+s+s2}{\PYZdq{}}\PY{l+s+s2}{File contains 1 invalid row!}\PY{l+s+s2}{\PYZdq{}}\PY{p}{)}
                    \PY{k}{else}\PY{p}{:}
                        \PY{n+nb}{print}\PY{p}{(}\PY{l+s+s2}{\PYZdq{}}\PY{l+s+s2}{File contains }\PY{l+s+s2}{\PYZdq{}}\PY{o}{+}\PY{n+nb}{str}\PY{p}{(}\PY{n}{invalidRows}\PY{p}{)}\PY{o}{+} \PY{l+s+s2}{\PYZdq{}}\PY{l+s+s2}{ invalid rows!}\PY{l+s+s2}{\PYZdq{}}\PY{p}{)}
        
                    \PY{c+c1}{\PYZsh{}make a new file}
                    \PY{k}{if} \PY{n+nb+bp}{self}\PY{o}{.}\PY{n}{fileName}\PY{p}{[}\PY{o}{\PYZhy{}}\PY{l+m+mi}{4}\PY{p}{:}\PY{p}{]} \PY{o}{==} \PY{l+s+s2}{\PYZdq{}}\PY{l+s+s2}{.csv}\PY{l+s+s2}{\PYZdq{}}\PY{p}{:}
                        \PY{n+nb}{print}\PY{p}{(}\PY{l+s+s2}{\PYZdq{}}\PY{l+s+s2}{Refining the data given and storing it a new file...}\PY{l+s+s2}{\PYZdq{}}\PY{p}{)}
                        \PY{n}{newFileName} \PY{o}{=} \PY{n+nb+bp}{self}\PY{o}{.}\PY{n}{fileName}\PY{p}{[}\PY{p}{:}\PY{o}{\PYZhy{}}\PY{l+m+mi}{4}\PY{p}{]} \PY{o}{+} \PY{l+s+s2}{\PYZdq{}}\PY{l+s+s2}{\PYZus{}refined.csv}\PY{l+s+s2}{\PYZdq{}}
                        \PY{n+nb+bp}{self}\PY{o}{.}\PY{n}{data}\PY{o}{.}\PY{n}{to\PYZus{}csv}\PY{p}{(}\PY{n}{newFileName}\PY{p}{)}
                        \PY{n+nb}{print}\PY{p}{(}\PY{l+s+s2}{\PYZdq{}}\PY{l+s+s2}{Refined data can be found in }\PY{l+s+se}{\PYZbs{}\PYZsq{}}\PY{l+s+s2}{\PYZdq{}} \PY{o}{+} \PY{n}{newFileName} \PY{o}{+} \PY{l+s+s2}{\PYZdq{}}\PY{l+s+se}{\PYZbs{}\PYZsq{}}\PY{l+s+s2}{.}\PY{l+s+s2}{\PYZdq{}}\PY{p}{)}
\end{Verbatim}


    Counting the numbers of tweets, replies, retweets, and users, and
calculating average user activity.

    \begin{Verbatim}[commandchars=\\\{\}]
{\color{incolor}In [{\color{incolor}5}]:} \PY{k+kn}{from} \PY{n+nn}{parsing} \PY{k}{import} \PY{o}{*}
        
        \PY{n}{cr} \PY{o}{=} \PY{n}{CensusReader}\PY{p}{(}\PY{l+s+s2}{\PYZdq{}}\PY{l+s+s2}{data/CometLanding\PYZus{}refined.csv}\PY{l+s+s2}{\PYZdq{}}\PY{p}{)}
        \PY{n}{retweets} \PY{o}{=} \PY{n}{cr}\PY{o}{.}\PY{n}{retweetCount}\PY{p}{(}\PY{p}{)}
        \PY{n}{replies} \PY{o}{=} \PY{n}{cr}\PY{o}{.}\PY{n}{replyCount}\PY{p}{(}\PY{p}{)}
        \PY{n}{tweets} \PY{o}{=} \PY{n}{cr}\PY{o}{.}\PY{n}{tweetCount}\PY{p}{(}\PY{p}{)}
        \PY{n}{users} \PY{o}{=} \PY{n}{cr}\PY{o}{.}\PY{n}{userCount}\PY{p}{(}\PY{p}{)}
        
        \PY{n+nb}{print}\PY{p}{(}\PY{l+s+s2}{\PYZdq{}}\PY{l+s+se}{\PYZbs{}n}\PY{l+s+s2}{There were }\PY{l+s+s2}{\PYZdq{}} \PY{o}{+} \PY{n+nb}{str}\PY{p}{(}\PY{n}{tweets}\PY{p}{)} \PY{o}{+} \PY{l+s+s2}{\PYZdq{}}\PY{l+s+s2}{ tweets in total.}\PY{l+s+s2}{\PYZdq{}}\PY{p}{)}
        \PY{n+nb}{print}\PY{p}{(}\PY{l+s+s2}{\PYZdq{}}\PY{l+s+s2}{There were }\PY{l+s+s2}{\PYZdq{}} \PY{o}{+} \PY{n+nb}{str}\PY{p}{(}\PY{n}{replies}\PY{p}{)} \PY{o}{+} \PY{l+s+s2}{\PYZdq{}}\PY{l+s+s2}{ replies in total.}\PY{l+s+s2}{\PYZdq{}}\PY{p}{)}
        \PY{n+nb}{print}\PY{p}{(}\PY{l+s+s2}{\PYZdq{}}\PY{l+s+s2}{There were }\PY{l+s+s2}{\PYZdq{}} \PY{o}{+} \PY{n+nb}{str}\PY{p}{(}\PY{n}{retweets}\PY{p}{)} \PY{o}{+} \PY{l+s+s2}{\PYZdq{}}\PY{l+s+s2}{ retweets in total.}\PY{l+s+se}{\PYZbs{}n}\PY{l+s+s2}{\PYZdq{}}\PY{p}{)}
        
        \PY{n+nb}{print}\PY{p}{(}\PY{l+s+s2}{\PYZdq{}}\PY{l+s+s2}{There were }\PY{l+s+s2}{\PYZdq{}} \PY{o}{+} \PY{n+nb}{str}\PY{p}{(}\PY{n}{users}\PY{p}{)} \PY{o}{+} \PY{l+s+s2}{\PYZdq{}}\PY{l+s+s2}{ users tweeting in the dataset.}\PY{l+s+se}{\PYZbs{}n}\PY{l+s+s2}{\PYZdq{}}\PY{p}{)}
        
        \PY{n+nb}{print}\PY{p}{(}\PY{l+s+s2}{\PYZdq{}}\PY{l+s+s2}{The average number of tweets by a user in the dataset was }\PY{l+s+s2}{\PYZdq{}} \PY{o}{+} \PY{n+nb}{str}\PY{p}{(}\PY{n}{tweets} \PY{o}{/} \PY{n}{users}\PY{p}{)} \PY{o}{+} \PY{l+s+s2}{\PYZdq{}}\PY{l+s+s2}{.}\PY{l+s+s2}{\PYZdq{}}\PY{p}{)}
        \PY{n+nb}{print}\PY{p}{(}\PY{l+s+s2}{\PYZdq{}}\PY{l+s+s2}{The average number of replies by a user in the dataset was }\PY{l+s+s2}{\PYZdq{}} \PY{o}{+} \PY{n+nb}{str}\PY{p}{(}\PY{n}{replies} \PY{o}{/} \PY{n}{users}\PY{p}{)} \PY{o}{+} \PY{l+s+s2}{\PYZdq{}}\PY{l+s+s2}{.}\PY{l+s+s2}{\PYZdq{}}\PY{p}{)}
        \PY{n+nb}{print}\PY{p}{(}\PY{l+s+s2}{\PYZdq{}}\PY{l+s+s2}{The average number of retweets by a user in the dataset was }\PY{l+s+s2}{\PYZdq{}} \PY{o}{+} \PY{n+nb}{str}\PY{p}{(}\PY{n}{retweets} \PY{o}{/} \PY{n}{users}\PY{p}{)} \PY{o}{+} \PY{l+s+s2}{\PYZdq{}}\PY{l+s+s2}{.}\PY{l+s+s2}{\PYZdq{}}\PY{p}{)}
\end{Verbatim}


    \begin{Verbatim}[commandchars=\\\{\}]

There were 15547 tweets in total.
There were 1723 replies in total.
There were 59998 retweets in total.

There were 50192 users tweeting in the dataset.

The average number of tweets by a user in the dataset was 0.309750557858.
The average number of replies by a user in the dataset was 0.0343281797896.
The average number of retweets by a user in the dataset was 1.1953697800446286.

    \end{Verbatim}

    retweetCount() method

The retweetCount() method first finds if the tweet text for each item in
the DataFrame from the CSV data starts with "RT", indicating it is a
retweet. Each result is added to a new list of Boolean values; the sum()
method is then used to count the number of True values in the list (i.e.
retweets). The value of the sum function is then returned.

    \begin{Verbatim}[commandchars=\\\{\}]
{\color{incolor}In [{\color{incolor}6}]:} \PY{k}{def} \PY{n+nf}{retweetCount}\PY{p}{(}\PY{n+nb+bp}{self}\PY{p}{)}\PY{p}{:}
            \PY{k}{return} \PY{p}{(}\PY{n+nb+bp}{self}\PY{o}{.}\PY{n}{data}\PY{p}{[}\PY{l+s+s1}{\PYZsq{}}\PY{l+s+s1}{text}\PY{l+s+s1}{\PYZsq{}}\PY{p}{]}\PY{o}{.}\PY{n}{str}\PY{o}{.}\PY{n}{startswith}\PY{p}{(}\PY{l+s+s2}{\PYZdq{}}\PY{l+s+s2}{RT}\PY{l+s+s2}{\PYZdq{}}\PY{p}{)}\PY{p}{)}\PY{o}{.}\PY{n}{sum}\PY{p}{(}\PY{p}{)}
\end{Verbatim}


    replyCount() method

The replyCount() method first finds if the 'in\_reply\_to\_screen\_name'
field for each item in the DataFrame from the CSV isn't empty,
indicating it is a reply. Each result is added to a new list of Boolean
values; the sum() method is then used to count the number of True values
in the list (i.e. replies). The value of the sum function is then
returned.

    \begin{Verbatim}[commandchars=\\\{\}]
{\color{incolor}In [{\color{incolor}7}]:} \PY{k}{def} \PY{n+nf}{replyCount}\PY{p}{(}\PY{n+nb+bp}{self}\PY{p}{)}\PY{p}{:}
            \PY{k}{return} \PY{p}{(}\PY{n+nb+bp}{self}\PY{o}{.}\PY{n}{data}\PY{p}{[}\PY{l+s+s1}{\PYZsq{}}\PY{l+s+s1}{in\PYZus{}reply\PYZus{}to\PYZus{}screen\PYZus{}name}\PY{l+s+s1}{\PYZsq{}}\PY{p}{]}\PY{o}{.}\PY{n}{notnull}\PY{p}{(}\PY{p}{)}\PY{p}{)}\PY{o}{.}\PY{n}{sum}\PY{p}{(}\PY{p}{)}
\end{Verbatim}


    tweetCount() method

The tweetCount() method first finds the number of unique tweets. It does
this by taking all parsable entries in the DataFrame (from which
duplicates have already been dropped). It then finds the length of the
resulting DataFrame (of all unique tweets), representing the total
number of tweets of any kind. To then find the number of tweets that are
not replies or retweets, we then subtract the result of the replyCount()
and retweetCount() methods from the number of unique tweets found.

    \begin{Verbatim}[commandchars=\\\{\}]
{\color{incolor}In [{\color{incolor}8}]:} \PY{k}{def} \PY{n+nf}{tweetCount}\PY{p}{(}\PY{n+nb+bp}{self}\PY{p}{)}\PY{p}{:}
            \PY{k}{return} \PY{n+nb}{len}\PY{p}{(}\PY{n+nb+bp}{self}\PY{o}{.}\PY{n}{data}\PY{p}{[}\PY{l+s+s1}{\PYZsq{}}\PY{l+s+s1}{entities\PYZus{}str}\PY{l+s+s1}{\PYZsq{}}\PY{p}{]}\PY{o}{.}\PY{n}{notnull}\PY{p}{(}\PY{p}{)}\PY{p}{)} \PY{o}{\PYZhy{}} \PY{n+nb+bp}{self}\PY{o}{.}\PY{n}{replyCount}\PY{p}{(}\PY{p}{)} \PY{o}{\PYZhy{}} \PY{n+nb+bp}{self}\PY{o}{.}\PY{n}{retweetCount}\PY{p}{(}\PY{p}{)}
\end{Verbatim}


    userCount() method

The userCount() method first finds entries in the DataFrane who have
unique posting users (i.e. removing all but one tweet posted by each
user with one or more tweet in the dataset) It does this with the pandas
drop\_duplicates() method. The length of the resulting DataFrame is then
returned, representing the number of unique uesers who posted tweets in
the dataset.

    \begin{Verbatim}[commandchars=\\\{\}]
{\color{incolor}In [{\color{incolor}9}]:} \PY{k}{def} \PY{n+nf}{userCount}\PY{p}{(}\PY{n+nb+bp}{self}\PY{p}{)}\PY{p}{:}
            \PY{k}{return} \PY{n+nb}{len}\PY{p}{(}\PY{n+nb+bp}{self}\PY{o}{.}\PY{n}{data}\PY{o}{.}\PY{n}{drop\PYZus{}duplicates}\PY{p}{(}\PY{n}{subset}\PY{o}{=}\PY{p}{[}\PY{l+s+s1}{\PYZsq{}}\PY{l+s+s1}{from\PYZus{}user\PYZus{}id\PYZus{}str}\PY{l+s+s1}{\PYZsq{}}\PY{p}{]}\PY{p}{)}\PY{p}{)}
\end{Verbatim}


    Pie chart showing composition of activity by type

    \begin{Verbatim}[commandchars=\\\{\}]
{\color{incolor}In [{\color{incolor}10}]:} \PY{k+kn}{from} \PY{n+nn}{parsing} \PY{k}{import} \PY{o}{*}
         \PY{k+kn}{from} \PY{n+nn}{plotter} \PY{k}{import} \PY{o}{*}
         
         \PY{n}{cr} \PY{o}{=} \PY{n}{CensusReader}\PY{p}{(}\PY{l+s+s2}{\PYZdq{}}\PY{l+s+s2}{data/CometLanding\PYZus{}refined.csv}\PY{l+s+s2}{\PYZdq{}}\PY{p}{)}
         
         \PY{n}{retweets} \PY{o}{=} \PY{n}{cr}\PY{o}{.}\PY{n}{retweetCount}\PY{p}{(}\PY{p}{)}
         \PY{n}{replies} \PY{o}{=} \PY{n}{cr}\PY{o}{.}\PY{n}{replyCount}\PY{p}{(}\PY{p}{)}
         \PY{n}{tweets} \PY{o}{=} \PY{n}{cr}\PY{o}{.}\PY{n}{tweetCount}\PY{p}{(}\PY{p}{)}
         
         \PY{n}{plotter} \PY{o}{=} \PY{n}{Plotter}\PY{p}{(}\PY{p}{)}
         \PY{n}{plotter}\PY{o}{.}\PY{n}{pieChart}\PY{p}{(}\PY{n}{tweets}\PY{p}{,} \PY{n}{retweets}\PY{p}{,} \PY{n}{replies}\PY{p}{)}
\end{Verbatim}


    \begin{center}
    \adjustimage{max size={0.9\linewidth}{0.9\paperheight}}{output_19_0.png}
    \end{center}
    { \hspace*{\fill} \\}
    
    pieChart() method

The pieChart() method begins by adding the numbers of tweets, retweets,
and replies in the DataFrame. It then finds the percentage proportion of
all activity for each activity type (tweets, retweets, and replies).It
then creates lists of labels for the legend, the "slice" sizes, and
colours of the pie chart. These and other arguments are then passed to
plt.pie() and other plt() methods to generate and then display the pie
chart (plt represents matplotlib.pyplot).

    \begin{Verbatim}[commandchars=\\\{\}]
{\color{incolor}In [{\color{incolor}11}]:} \PY{k}{def} \PY{n+nf}{pieChart}\PY{p}{(}\PY{n+nb+bp}{self}\PY{p}{,} \PY{n}{tweets}\PY{p}{,} \PY{n}{retweets}\PY{p}{,} \PY{n}{replies}\PY{p}{)}\PY{p}{:}
             \PY{n}{total} \PY{o}{=} \PY{n}{tweets} \PY{o}{+} \PY{n}{retweets} \PY{o}{+} \PY{n}{replies}
             \PY{c+c1}{\PYZsh{} The slices will be ordered and plotted counter\PYZhy{}clockwise.}
             \PY{n}{percents} \PY{o}{=} \PY{p}{[}\PY{p}{(}\PY{n+nb}{float}\PY{p}{(}\PY{n}{tweets}\PY{p}{)}\PY{o}{/}\PY{n+nb}{float}\PY{p}{(}\PY{n}{total}\PY{p}{)}\PY{o}{*}\PY{l+m+mf}{100.0}\PY{p}{)}\PY{p}{,} \PY{p}{(}\PY{n+nb}{float}\PY{p}{(}\PY{n}{retweets}\PY{p}{)}\PY{o}{/}\PY{n+nb}{float}\PY{p}{(}\PY{n}{total}\PY{p}{)}\PY{o}{*}\PY{l+m+mf}{100.0}\PY{p}{)}\PY{p}{,}\PY{p}{(}\PY{n+nb}{float}\PY{p}{(}\PY{n}{replies}\PY{p}{)}\PY{o}{/}\PY{n+nb}{float}\PY{p}{(}\PY{n}{total}\PY{p}{)}\PY{o}{*}\PY{l+m+mf}{100.0}\PY{p}{)}\PY{p}{]}
         
             \PY{n}{labels} \PY{o}{=} \PY{l+s+s1}{\PYZsq{}}\PY{l+s+s1}{Tweets (}\PY{l+s+s1}{\PYZsq{}} \PY{o}{+} \PY{n+nb}{str}\PY{p}{(}\PY{n}{percents}\PY{p}{[}\PY{l+m+mi}{0}\PY{p}{]}\PY{p}{)} \PY{o}{+} \PY{l+s+s1}{\PYZsq{}}\PY{l+s+s1}{\PYZpc{}}\PY{l+s+s1}{)}\PY{l+s+s1}{\PYZsq{}}\PY{p}{,} \PY{l+s+s1}{\PYZsq{}}\PY{l+s+s1}{Retweets (}\PY{l+s+s1}{\PYZsq{}} \PY{o}{+} \PY{n+nb}{str}\PY{p}{(}\PY{n}{percents}\PY{p}{[}\PY{l+m+mi}{1}\PY{p}{]}\PY{p}{)} \PY{o}{+} \PY{l+s+s1}{\PYZsq{}}\PY{l+s+s1}{\PYZpc{}}\PY{l+s+s1}{)}\PY{l+s+s1}{\PYZsq{}}\PY{p}{,} \PY{l+s+s1}{\PYZsq{}}\PY{l+s+s1}{Replies (}\PY{l+s+s1}{\PYZsq{}} \PY{o}{+} \PY{n+nb}{str}\PY{p}{(}\PY{n}{percents}\PY{p}{[}\PY{l+m+mi}{2}\PY{p}{]}\PY{p}{)} \PY{o}{+} \PY{l+s+s1}{\PYZsq{}}\PY{l+s+s1}{\PYZpc{}}\PY{l+s+s1}{)}\PY{l+s+s1}{\PYZsq{}}
             \PY{n}{fracs} \PY{o}{=} \PY{p}{[}\PY{n}{tweets}\PY{p}{,} \PY{n}{retweets}\PY{p}{,} \PY{n}{replies}\PY{p}{]}
             \PY{n}{colors} \PY{o}{=} \PY{p}{[}\PY{l+s+s1}{\PYZsq{}}\PY{l+s+s1}{\PYZsh{}ff4d4d}\PY{l+s+s1}{\PYZsq{}}\PY{p}{,} \PY{l+s+s1}{\PYZsq{}}\PY{l+s+s1}{\PYZsh{}0FAC36}\PY{l+s+s1}{\PYZsq{}}\PY{p}{,} \PY{l+s+s1}{\PYZsq{}}\PY{l+s+s1}{\PYZsh{}0F18AC}\PY{l+s+s1}{\PYZsq{}}\PY{p}{]} \PY{c+c1}{\PYZsh{} red, green, blue}
         
             \PY{n}{patches}\PY{p}{,} \PY{n}{texts} \PY{o}{=} \PY{n}{plt}\PY{o}{.}\PY{n}{pie}\PY{p}{(}\PY{n}{fracs}\PY{p}{,} \PY{n}{colors}\PY{o}{=}\PY{n}{colors}\PY{p}{,} \PY{n}{shadow}\PY{o}{=}\PY{k+kc}{True}\PY{p}{,} \PY{n}{startangle}\PY{o}{=}\PY{l+m+mi}{90}\PY{p}{)}
             \PY{n}{plt}\PY{o}{.}\PY{n}{legend}\PY{p}{(}\PY{n}{patches}\PY{p}{,} \PY{n}{labels}\PY{p}{,} \PY{n}{loc}\PY{o}{=}\PY{l+s+s2}{\PYZdq{}}\PY{l+s+s2}{best}\PY{l+s+s2}{\PYZdq{}}\PY{p}{)}
             \PY{n}{plt}\PY{o}{.}\PY{n}{axis}\PY{p}{(}\PY{l+s+s1}{\PYZsq{}}\PY{l+s+s1}{equal}\PY{l+s+s1}{\PYZsq{}}\PY{p}{)}
         
             \PY{n}{plt}\PY{o}{.}\PY{n}{tight\PYZus{}layout}\PY{p}{(}\PY{p}{)}
             \PY{n}{plt}\PY{o}{.}\PY{n}{title}\PY{p}{(}\PY{l+s+s1}{\PYZsq{}}\PY{l+s+s1}{Pie Chart Showing Frequencies of Different Activity Types}\PY{l+s+s1}{\PYZsq{}}\PY{p}{)}
             \PY{n}{plt}\PY{o}{.}\PY{n}{show}\PY{p}{(}\PY{p}{)}
\end{Verbatim}


    Bar chart showing composition of activity by type

    \begin{Verbatim}[commandchars=\\\{\}]
{\color{incolor}In [{\color{incolor}12}]:} \PY{k+kn}{from} \PY{n+nn}{parsing} \PY{k}{import} \PY{o}{*}
         \PY{k+kn}{from} \PY{n+nn}{plotter} \PY{k}{import} \PY{o}{*}
         
         \PY{n}{cr} \PY{o}{=} \PY{n}{CensusReader}\PY{p}{(}\PY{l+s+s2}{\PYZdq{}}\PY{l+s+s2}{data/CometLanding\PYZus{}refined.csv}\PY{l+s+s2}{\PYZdq{}}\PY{p}{)}
         
         \PY{n}{retweets} \PY{o}{=} \PY{n}{cr}\PY{o}{.}\PY{n}{retweetCount}\PY{p}{(}\PY{p}{)}
         \PY{n}{replies} \PY{o}{=} \PY{n}{cr}\PY{o}{.}\PY{n}{replyCount}\PY{p}{(}\PY{p}{)}
         \PY{n}{tweets} \PY{o}{=} \PY{n}{cr}\PY{o}{.}\PY{n}{tweetCount}\PY{p}{(}\PY{p}{)}
         
         
         \PY{n}{plotter} \PY{o}{=} \PY{n}{Plotter}\PY{p}{(}\PY{p}{)}
         \PY{n}{plotter}\PY{o}{.}\PY{n}{barChart}\PY{p}{(}\PY{n}{tweets}\PY{p}{,} \PY{n}{retweets}\PY{p}{,} \PY{n}{replies}\PY{p}{)}
\end{Verbatim}


    \begin{center}
    \adjustimage{max size={0.9\linewidth}{0.9\paperheight}}{output_23_0.png}
    \end{center}
    { \hspace*{\fill} \\}
    
    The barChart() method

The barChart() method is very similar to the pieChart() method, using
the numbers of tweets, retweets, and replies in the DataFrame. Again,
values to be used as parameters for the bar chart are generated from the
fequency of each activity type; these are used with their Names, as well
as axis labels and a title to generate the bar chart, again with plt as
matplotlib.pyplot.

    \begin{Verbatim}[commandchars=\\\{\}]
{\color{incolor}In [{\color{incolor}13}]:} \PY{k}{def} \PY{n+nf}{barChart}\PY{p}{(}\PY{n+nb+bp}{self}\PY{p}{,} \PY{n}{tweets}\PY{p}{,} \PY{n}{retweets}\PY{p}{,} \PY{n}{replies}\PY{p}{)}\PY{p}{:}
             \PY{n}{vals} \PY{o}{=} \PY{p}{[}\PY{n}{tweets}\PY{p}{,} \PY{n}{retweets}\PY{p}{,} \PY{n}{replies}\PY{p}{]}
             \PY{n}{labels} \PY{o}{=} \PY{p}{(}\PY{l+s+s1}{\PYZsq{}}\PY{l+s+s1}{Tweets}\PY{l+s+s1}{\PYZsq{}}\PY{p}{,} \PY{l+s+s1}{\PYZsq{}}\PY{l+s+s1}{Retweets}\PY{l+s+s1}{\PYZsq{}}\PY{p}{,} \PY{l+s+s1}{\PYZsq{}}\PY{l+s+s1}{Replies}\PY{l+s+s1}{\PYZsq{}}\PY{p}{)}
             \PY{n}{n\PYZus{}groups} \PY{o}{=} \PY{n+nb}{len}\PY{p}{(}\PY{n}{vals}\PY{p}{)}
             \PY{n}{bar\PYZus{}width} \PY{o}{=} \PY{l+m+mi}{1}\PY{o}{/}\PY{l+m+mf}{1.5}
             \PY{n}{pos} \PY{o}{=} \PY{n}{np}\PY{o}{.}\PY{n}{arange}\PY{p}{(}\PY{n+nb}{len}\PY{p}{(}\PY{n}{labels}\PY{p}{)}\PY{p}{)}
             
             \PY{n}{plt}\PY{o}{.}\PY{n}{bar}\PY{p}{(}\PY{n}{pos}\PY{p}{,} \PY{n}{vals}\PY{p}{,} \PY{n}{align}\PY{o}{=}\PY{l+s+s1}{\PYZsq{}}\PY{l+s+s1}{center}\PY{l+s+s1}{\PYZsq{}}\PY{p}{,} \PY{n}{alpha}\PY{o}{=}\PY{l+m+mf}{0.5}\PY{p}{)}
             \PY{n}{plt}\PY{o}{.}\PY{n}{xticks}\PY{p}{(}\PY{n}{pos}\PY{p}{,} \PY{n}{labels}\PY{p}{)}
             \PY{n}{plt}\PY{o}{.}\PY{n}{ylabel}\PY{p}{(}\PY{l+s+s1}{\PYZsq{}}\PY{l+s+s1}{Absolute Frequency}\PY{l+s+s1}{\PYZsq{}}\PY{p}{)}
             \PY{n}{plt}\PY{o}{.}\PY{n}{xlabel}\PY{p}{(}\PY{l+s+s1}{\PYZsq{}}\PY{l+s+s1}{Activity Type}\PY{l+s+s1}{\PYZsq{}}\PY{p}{)}
             \PY{n}{plt}\PY{o}{.}\PY{n}{title}\PY{p}{(}\PY{l+s+s1}{\PYZsq{}}\PY{l+s+s1}{Bar Chart Showing Frequencies of Different Activity Types}\PY{l+s+s1}{\PYZsq{}}\PY{p}{)}
             
             \PY{n}{plt}\PY{o}{.}\PY{n}{show}\PY{p}{(}\PY{p}{)}
\end{Verbatim}


    Finding the 10 most common hashtags used

    \begin{Verbatim}[commandchars=\\\{\}]
{\color{incolor}In [{\color{incolor}14}]:} \PY{k+kn}{from} \PY{n+nn}{parsing} \PY{k}{import} \PY{o}{*}
         
         \PY{n}{cr} \PY{o}{=} \PY{n}{CensusReader}\PY{p}{(}\PY{l+s+s2}{\PYZdq{}}\PY{l+s+s2}{data/CometLanding\PYZus{}refined.csv}\PY{l+s+s2}{\PYZdq{}}\PY{p}{)}
         \PY{n+nb}{print}\PY{p}{(}\PY{n}{cr}\PY{o}{.}\PY{n}{mostPopHashtags}\PY{p}{(}\PY{l+m+mi}{10}\PY{p}{,} \PY{k+kc}{False}\PY{p}{)}\PY{p}{)}
         \PY{n+nb}{print}\PY{p}{(}\PY{l+s+s2}{\PYZdq{}}\PY{l+s+se}{\PYZbs{}n}\PY{l+s+se}{\PYZbs{}n}\PY{l+s+s2}{\PYZdq{}}\PY{p}{)}
         \PY{n+nb}{print}\PY{p}{(}\PY{n}{cr}\PY{o}{.}\PY{n}{mostPopHashtags}\PY{p}{(}\PY{l+m+mi}{10}\PY{p}{,} \PY{k+kc}{True}\PY{p}{)}\PY{p}{)}
\end{Verbatim}


    \begin{Verbatim}[commandchars=\\\{\}]
1. \#CometLanding : 62752 tweets
2. \#cometlanding : 13211 tweets
3. \#67P : 7922 tweets
4. \#Rosetta : 5912 tweets
5. \#Philae : 3097 tweets
6. \#Cometlanding : 1036 tweets
7. \#WishKoSaPasko : 988 tweets
8. \#HappyBirthdaySandaraPark : 965 tweets
9. \#rosetta : 702 tweets
10. \#esa : 680 tweets




1. \#67P : 7922 tweets
2. \#Rosetta : 5912 tweets
3. \#Philae : 3097 tweets
4. \#WishKoSaPasko : 988 tweets
5. \#HappyBirthdaySandaraPark : 965 tweets
6. \#rosetta : 702 tweets
7. \#esa : 680 tweets
8. \#SEP : 623 tweets
9. \#philae : 600 tweets
10. \#PoseToiPhilae : 569 tweets


    \end{Verbatim}

    The mostPopHashtags() method

The mostPopHashtags() method begins by getting sorted list of hashtags
and their numbers of occurances from the getHTagsAndCounts() method. The
method then checks if the user wants to exclude variants of the
\#CometLanding hashtag from the results; if so, the method will iterate
over the items in the list up to the number to be printed, removing any
matches to the \#CometLanding hashtag and then moving back. The method
then iterates over the resultant list of hashtags up to the nth largest
to be printed, printing each hashtag and the number of tweets in the
DataFrame it appears in.

    \begin{Verbatim}[commandchars=\\\{\}]
{\color{incolor}In [{\color{incolor}15}]:} \PY{k}{def} \PY{n+nf}{mostPopHashtags}\PY{p}{(}\PY{n+nb+bp}{self}\PY{p}{,} \PY{n}{number}\PY{p}{,} \PY{n}{excludeCometLanding}\PY{p}{)}\PY{p}{:}
             \PY{n}{sortedHTagsAndCounts} \PY{o}{=} \PY{n+nb}{sorted}\PY{p}{(}\PY{n+nb+bp}{self}\PY{o}{.}\PY{n}{getHTagsAndCounts}\PY{p}{(}\PY{p}{)}\PY{p}{,} \PY{n}{key} \PY{o}{=} \PY{n}{itemgetter}\PY{p}{(}\PY{l+m+mi}{1}\PY{p}{)}\PY{p}{,} \PY{n}{reverse}\PY{o}{=}\PY{k+kc}{True}\PY{p}{)}
             \PY{n}{returnString} \PY{o}{=} \PY{l+s+s2}{\PYZdq{}}\PY{l+s+s2}{\PYZdq{}}
             
             \PY{k}{if} \PY{n}{excludeCometLanding}\PY{p}{:}
                 \PY{n+nb}{max} \PY{o}{=} \PY{n}{number} \PY{o}{\PYZhy{}} \PY{l+m+mi}{1}
                 \PY{n}{x} \PY{o}{=} \PY{l+m+mi}{0}
                 \PY{k}{while} \PY{n}{x} \PY{o}{\PYZlt{}} \PY{n+nb}{max}\PY{p}{:}
                     \PY{k}{if} \PY{n}{sortedHTagsAndCounts}\PY{p}{[}\PY{n}{x}\PY{p}{]}\PY{p}{[}\PY{l+m+mi}{0}\PY{p}{]}\PY{o}{.}\PY{n}{lower}\PY{p}{(}\PY{p}{)} \PY{o}{==} \PY{l+s+s2}{\PYZdq{}}\PY{l+s+s2}{cometlanding}\PY{l+s+s2}{\PYZdq{}}\PY{p}{:}
                         \PY{k}{del} \PY{n}{sortedHTagsAndCounts}\PY{p}{[}\PY{n}{x}\PY{p}{]}
                     \PY{k}{else}\PY{p}{:} \PY{n}{x} \PY{o}{+}\PY{o}{=} \PY{l+m+mi}{1}
             
             \PY{k}{for} \PY{n}{x} \PY{o+ow}{in} \PY{n+nb}{range}\PY{p}{(}\PY{l+m+mi}{0}\PY{p}{,}\PY{n}{number}\PY{p}{)}\PY{p}{:}
                 \PY{n}{returnString} \PY{o}{+}\PY{o}{=} \PY{n+nb}{str}\PY{p}{(}\PY{n}{x} \PY{o}{+} \PY{l+m+mi}{1}\PY{p}{)} \PY{o}{+} \PY{l+s+s2}{\PYZdq{}}\PY{l+s+s2}{. \PYZsh{}}\PY{l+s+s2}{\PYZdq{}} \PY{o}{+} \PY{n}{sortedHTagsAndCounts}\PY{p}{[}\PY{n}{x}\PY{p}{]}\PY{p}{[}\PY{l+m+mi}{0}\PY{p}{]} \PY{o}{+} \PY{l+s+s2}{\PYZdq{}}\PY{l+s+s2}{ : }\PY{l+s+s2}{\PYZdq{}} \PY{o}{+} \PY{n+nb}{str}\PY{p}{(}\PY{n}{sortedHTagsAndCounts}\PY{p}{[}\PY{n}{x}\PY{p}{]}\PY{p}{[}\PY{l+m+mi}{1}\PY{p}{]}\PY{p}{)} \PY{o}{+} \PY{l+s+s2}{\PYZdq{}}\PY{l+s+s2}{ tweets}\PY{l+s+se}{\PYZbs{}n}\PY{l+s+s2}{\PYZdq{}}
             
             \PY{k}{return} \PY{n}{returnString}
\end{Verbatim}


    The getHTagsAndCounts() method

The getHTagsAndCounts() method begins by getting a list of every
hashtag; each is present in the list the number of times it is tweeted
as hashtags are extracted from tweets and added to the list for each
tweet. This is done in the getHashTags() method. This list is then
sorted, before a list without duplicates is found using the
removeDuplicates() method. This new list without duplicates is then
iterated over; each hashtag in it, and the number of times it is present
in the original list of hashtags (found using a Counter) is added in a
new array item in a new 2D array, hTagsAndCounts, containing each
hashtag, and the number of times it is present in a tweet in the
DataFrame. This new 2D array is then returned.

    \begin{Verbatim}[commandchars=\\\{\}]
{\color{incolor}In [{\color{incolor}16}]:} \PY{k}{def} \PY{n+nf}{getHTagsAndCounts}\PY{p}{(}\PY{n+nb+bp}{self}\PY{p}{)}\PY{p}{:}
             \PY{n}{hashtags} \PY{o}{=} \PY{n+nb}{sorted}\PY{p}{(}\PY{n+nb+bp}{self}\PY{o}{.}\PY{n}{getHashTags}\PY{p}{(}\PY{p}{)}\PY{p}{,} \PY{n}{reverse}\PY{o}{=}\PY{k+kc}{False}\PY{p}{)}
             \PY{n}{hashtagsNoDuplicates} \PY{o}{=} \PY{n+nb+bp}{self}\PY{o}{.}\PY{n}{removeDuplicates}\PY{p}{(}\PY{n}{hashtags}\PY{p}{)}
             \PY{n}{hTagsAndCounts} \PY{o}{=} \PY{p}{[}\PY{p}{]}
             \PY{n}{counter} \PY{o}{=} \PY{n}{Counter}\PY{p}{(}\PY{n}{hashtags}\PY{p}{)}
             \PY{k}{for} \PY{n}{ht} \PY{o+ow}{in} \PY{n}{hashtagsNoDuplicates}\PY{p}{:}
                 \PY{n}{count} \PY{o}{=} \PY{n}{counter}\PY{p}{[}\PY{n}{ht}\PY{p}{]} \PY{c+c1}{\PYZsh{} hashtags.count(ht)}
                 \PY{n}{hTagsAndCounts}\PY{o}{.}\PY{n}{append}\PY{p}{(}\PY{p}{[}\PY{n}{ht}\PY{p}{,} \PY{n}{count}\PY{p}{]}\PY{p}{)}
             \PY{k}{return} \PY{n}{hTagsAndCounts}
\end{Verbatim}


    The getHashTags() method

The getHashTags() method generates a list of each hashtag each time it
is in a tweet in the DataFrame. It begins by removing all fields but
'entities\_str' for each entry in the DataFrame using the loc pandas
function. It then generates a JSON object from the 'entities\_str' field
for each entry in the DataFrame using json.load(). The JSON array of
hashtags are then extracted from this, and iterated over; each hashtag
in the JSON hashtag list in the JSON entities object for each entry
(tweet) is also then iterated over. The text of each hashtag is then
appended to the list of hashtags being build. When all iteration
terminates, the resultant list of hashtags is returned.

    \begin{Verbatim}[commandchars=\\\{\}]
{\color{incolor}In [{\color{incolor}17}]:} \PY{k}{def} \PY{n+nf}{getHashTags}\PY{p}{(}\PY{n+nb+bp}{self}\PY{p}{)}\PY{p}{:}
             \PY{c+c1}{\PYZsh{}     hashtags = self.data.groupby(\PYZsq{}entities\PYZus{}str\PYZsq{}).count().sort\PYZus{}values(by=[\PYZsq{}id\PYZus{}str\PYZsq{}], ascending=False).head(number)}
             \PY{n}{data} \PY{o}{=} \PY{n+nb+bp}{self}\PY{o}{.}\PY{n}{data}
             \PY{n}{hashtags} \PY{o}{=} \PY{p}{[}\PY{p}{]}
             \PY{n}{jsonItems} \PY{o}{=} \PY{n}{data}\PY{o}{.}\PY{n}{loc}\PY{p}{[}\PY{p}{:}\PY{p}{,} \PY{l+s+s1}{\PYZsq{}}\PY{l+s+s1}{entities\PYZus{}str}\PY{l+s+s1}{\PYZsq{}}\PY{p}{]} \PY{c+c1}{\PYZsh{} lst of JSON hashtag items}
             \PY{c+c1}{\PYZsh{} each JSON hashtag item has syntax: \PYZob{}\PYZdq{}\PYZdq{}hashtags\PYZdq{}\PYZdq{}:[\PYZob{}\PYZdq{}\PYZdq{}text\PYZdq{}\PYZdq{}:\PYZdq{}\PYZdq{}67P\PYZdq{}\PYZdq{},\PYZdq{}\PYZdq{}indices\PYZdq{}\PYZdq{}:[58,62]\PYZcb{},\PYZob{}\PYZdq{}\PYZdq{}text\PYZdq{}\PYZdq{}:\PYZdq{}\PYZdq{}CometWatch\PYZdq{}\PYZdq{},\PYZdq{}\PYZdq{}indices\PYZdq{}\PYZdq{}:[127,138]\PYZcb{},\PYZob{}\PYZdq{}\PYZdq{}text\PYZdq{}\PYZdq{}:\PYZdq{}\PYZdq{}CometLanding\PYZdq{}\PYZdq{},\PYZdq{}\PYZdq{}indices\PYZdq{}\PYZdq{}:[139,140]\PYZcb{}],\PYZdq{}\PYZdq{}symbols\PYZdq{}\PYZdq{}:[],\PYZdq{}\PYZdq{}user\PYZus{}mentions\PYZdq{}\PYZdq{}:[\PYZob{}\PYZdq{}\PYZdq{}screen\PYZus{}name\PYZdq{}\PYZdq{}:\PYZdq{}\PYZdq{}ESA\PYZus{}Rosetta\PYZdq{}\PYZdq{},\PYZdq{}\PYZdq{}name\PYZdq{}\PYZdq{}:\PYZdq{}\PYZdq{}ESA Rosetta Mission\PYZdq{}\PYZdq{},\PYZdq{}\PYZdq{}id\PYZdq{}\PYZdq{}:253536357,\PYZdq{}\PYZdq{}id\PYZus{}str\PYZdq{}\PYZdq{}:\PYZdq{}\PYZdq{}253536357\PYZdq{}\PYZdq{},\PYZdq{}\PYZdq{}indices\PYZdq{}\PYZdq{}:[3,15]\PYZcb{}],\PYZdq{}\PYZdq{}urls\PYZdq{}\PYZdq{}:[\PYZob{}\PYZdq{}\PYZdq{}url\PYZdq{}\PYZdq{}:\PYZdq{}\PYZdq{}http://t.co/Z2A14Jorv6\PYZdq{}\PYZdq{},\PYZdq{}\PYZdq{}expanded\PYZus{}url\PYZdq{}\PYZdq{}:\PYZdq{}\PYZdq{}http://youtu.be/4a3eY5siRRk\PYZdq{}\PYZdq{},\PYZdq{}\PYZdq{}display\PYZus{}url\PYZdq{}\PYZdq{}:\PYZdq{}\PYZdq{}youtu.be/4a3eY5siRRk\PYZdq{}\PYZdq{},\PYZdq{}\PYZdq{}indices\PYZdq{}\PYZdq{}:[104,126]\PYZcb{}]\PYZcb{}\PYZdq{}.}
             \PY{k}{for} \PY{n}{JSONString} \PY{o+ow}{in} \PY{n}{jsonItems}\PY{p}{:}
                 \PY{k}{if} \PY{o+ow}{not} \PY{n+nb}{type}\PY{p}{(}\PY{n}{JSONString}\PY{p}{)} \PY{o+ow}{is} \PY{n+nb}{str}\PY{p}{:} \PY{k}{continue}
                 \PY{n}{parsedJSON} \PY{o}{=} \PY{n}{json}\PY{o}{.}\PY{n}{loads}\PY{p}{(}\PY{n}{JSONString}\PY{p}{)}
                 \PY{n}{JSONHashTagsElement} \PY{o}{=} \PY{n}{parsedJSON}\PY{p}{[}\PY{l+s+s1}{\PYZsq{}}\PY{l+s+s1}{hashtags}\PY{l+s+s1}{\PYZsq{}}\PY{p}{]}
                 \PY{k}{for} \PY{n}{JSONHashTag} \PY{o+ow}{in} \PY{n}{JSONHashTagsElement}\PY{p}{:}
                     \PY{n}{hashtags}\PY{o}{.}\PY{n}{append}\PY{p}{(}\PY{n}{JSONHashTag}\PY{p}{[}\PY{l+s+s1}{\PYZsq{}}\PY{l+s+s1}{text}\PY{l+s+s1}{\PYZsq{}}\PY{p}{]}\PY{p}{)}
             \PY{k}{return} \PY{n}{hashtags}
\end{Verbatim}


    The removeDuplicates() method

Copied from https://www.peterbe.com/plog/uniqifiers-benchmark. This
method adds all the items in the list seq to a dictionary as keys; if an
identical item is added it will override the previous entry for the
identical key, thus eliminating duplicates. The keyset from the
dictionary, copied from the seq list, is then returned.

    \begin{Verbatim}[commandchars=\\\{\}]
{\color{incolor}In [{\color{incolor}18}]:} \PY{k}{def} \PY{n+nf}{removeDuplicates}\PY{p}{(}\PY{n+nb+bp}{self}\PY{p}{,} \PY{n}{seq}\PY{p}{)}\PY{p}{:}
             \PY{c+c1}{\PYZsh{} Not order preserving}
             \PY{n}{keys} \PY{o}{=} \PY{p}{\PYZob{}}\PY{p}{\PYZcb{}}
             \PY{k}{for} \PY{n}{e} \PY{o+ow}{in} \PY{n}{seq}\PY{p}{:}
                 \PY{n}{keys}\PY{p}{[}\PY{n}{e}\PY{p}{]} \PY{o}{=} \PY{l+m+mi}{1}
             \PY{k}{return} \PY{n}{keys}\PY{o}{.}\PY{n}{keys}\PY{p}{(}\PY{p}{)}
\end{Verbatim}


    Hashtag wordcloud

    Hashtag wordcloud

The wordcloud code first filters all but the JSON entities field from
each entry in the DataSet. Each entities field (for each entry) is then
iterated over, and is parsed into a JSON object. Each hashtag in the
hashtags object in the entites JSON object is then iterated over. If the
text (field) of the hashtag is not "CometLanding" in some form, it is
added to the list of hashtags being built. The WordCloud function from
the wordcloud library is then used to build up a wordcloud image, and
plt (as matplotlib.pyplot) is then used to display the wordcloud.

    \begin{Verbatim}[commandchars=\\\{\}]
{\color{incolor}In [{\color{incolor}19}]:} \PY{k+kn}{from} \PY{n+nn}{parsing} \PY{k}{import} \PY{o}{*}
         \PY{k+kn}{import} \PY{n+nn}{json}
         \PY{k+kn}{from} \PY{n+nn}{wordcloud} \PY{k}{import} \PY{n}{WordCloud}
         \PY{k+kn}{import} \PY{n+nn}{matplotlib}\PY{n+nn}{.}\PY{n+nn}{pyplot} \PY{k}{as} \PY{n+nn}{plt}
         
         \PY{n}{data} \PY{o}{=} \PY{n}{CensusReader}\PY{p}{(}\PY{l+s+s2}{\PYZdq{}}\PY{l+s+s2}{data/CometLanding\PYZus{}refined.csv}\PY{l+s+s2}{\PYZdq{}}\PY{p}{)}\PY{o}{.}\PY{n}{data}
         \PY{n}{hashTags} \PY{o}{=} \PY{n}{data}\PY{o}{.}\PY{n}{loc}\PY{p}{[}\PY{p}{:}\PY{p}{,} \PY{l+s+s1}{\PYZsq{}}\PY{l+s+s1}{entities\PYZus{}str}\PY{l+s+s1}{\PYZsq{}}\PY{p}{]}
         \PY{n}{words} \PY{o}{=} \PY{p}{[}\PY{p}{]}
         \PY{k}{for} \PY{n}{i}\PY{p}{,} \PY{n}{v} \PY{o+ow}{in} \PY{n}{hashTags}\PY{o}{.}\PY{n}{iteritems}\PY{p}{(}\PY{p}{)}\PY{p}{:}
             \PY{k}{try}\PY{p}{:}
                 \PY{n}{j} \PY{o}{=} \PY{n}{json}\PY{o}{.}\PY{n}{loads}\PY{p}{(}\PY{n}{v}\PY{p}{)}
                 \PY{k}{for} \PY{n}{tag} \PY{o+ow}{in} \PY{n}{j}\PY{p}{[}\PY{l+s+s1}{\PYZsq{}}\PY{l+s+s1}{hashtags}\PY{l+s+s1}{\PYZsq{}}\PY{p}{]}\PY{p}{:}
                     \PY{k}{if} \PY{n}{tag}\PY{p}{[}\PY{l+s+s1}{\PYZsq{}}\PY{l+s+s1}{text}\PY{l+s+s1}{\PYZsq{}}\PY{p}{]}\PY{o}{.}\PY{n}{lower}\PY{p}{(}\PY{p}{)} \PY{o}{!=} \PY{l+s+s2}{\PYZdq{}}\PY{l+s+s2}{cometlanding}\PY{l+s+s2}{\PYZdq{}}\PY{p}{:}
                         \PY{n}{words}\PY{o}{.}\PY{n}{append}\PY{p}{(}\PY{n}{tag}\PY{p}{[}\PY{l+s+s1}{\PYZsq{}}\PY{l+s+s1}{text}\PY{l+s+s1}{\PYZsq{}}\PY{p}{]}\PY{p}{)}
             \PY{k}{except} \PY{n+ne}{TypeError}\PY{p}{:}
                 \PY{k}{pass}
         
         \PY{n}{wordcloud} \PY{o}{=} \PY{n}{WordCloud}\PY{p}{(}\PY{n}{width}\PY{o}{=}\PY{l+m+mi}{1000}\PY{p}{,} \PY{n}{height}\PY{o}{=}\PY{l+m+mi}{500}\PY{p}{)}\PY{o}{.}\PY{n}{generate}\PY{p}{(}\PY{l+s+s2}{\PYZdq{}}\PY{l+s+s2}{ }\PY{l+s+s2}{\PYZdq{}}\PY{o}{.}\PY{n}{join}\PY{p}{(}\PY{n}{words}\PY{p}{)}\PY{p}{)}
         \PY{n}{plt}\PY{o}{.}\PY{n}{figure}\PY{p}{(}\PY{p}{)}
         \PY{n}{plt}\PY{o}{.}\PY{n}{imshow}\PY{p}{(}\PY{n}{wordcloud}\PY{p}{,} \PY{n}{interpolation}\PY{o}{=}\PY{l+s+s2}{\PYZdq{}}\PY{l+s+s2}{bilinear}\PY{l+s+s2}{\PYZdq{}}\PY{p}{)}
         \PY{n}{plt}\PY{o}{.}\PY{n}{axis}\PY{p}{(}\PY{l+s+s2}{\PYZdq{}}\PY{l+s+s2}{off}\PY{l+s+s2}{\PYZdq{}}\PY{p}{)}
         \PY{n}{plt}\PY{o}{.}\PY{n}{show}\PY{p}{(}\PY{p}{)}
\end{Verbatim}


    \begin{center}
    \adjustimage{max size={0.9\linewidth}{0.9\paperheight}}{output_38_0.png}
    \end{center}
    { \hspace*{\fill} \\}
    
    Most common platforms

    \begin{Verbatim}[commandchars=\\\{\}]
{\color{incolor}In [{\color{incolor}20}]:} \PY{k+kn}{from} \PY{n+nn}{parsing} \PY{k}{import} \PY{o}{*}
         
         \PY{n}{cr} \PY{o}{=} \PY{n}{CensusReader}\PY{p}{(}\PY{l+s+s2}{\PYZdq{}}\PY{l+s+s2}{data/CometLanding\PYZus{}refined.csv}\PY{l+s+s2}{\PYZdq{}}\PY{p}{)}
         \PY{n+nb}{print} \PY{p}{(}\PY{n}{cr}\PY{o}{.}\PY{n}{appUsed}\PY{p}{(}\PY{p}{)}\PY{p}{)}
\end{Verbatim}


    \begin{Verbatim}[commandchars=\\\{\}]
Top Platforms used 
1. 'Twitter Web Client', 27925
2. 'Twitter for iPhone', 13743
3. 'Twitter for Android', 12770
4. 'TweetDeck', 4075
5. 'Twitter for iPad', 3282
6. 'dlvr.it', 1671
7. 'Twitter for Websites', 1462
8. 'Tweetbot for iΟS', 1055
9. 'Twitter for Windows Phone', 932
10. 'Tweet Old Post', 925
11. 'Twitter for Mac', 915
12. 'Twitter for Android Tablets', 897
13. 'Hootsuite', 796
14. 'TweetCaster for Android', 477
15. 'اشترركـــ الآن', 356
16. 'Twitter for BlackBerry®', 345
17. 'Mobile Web M5', 315
18. 'IFTTT', 303
19. 'Twitter for BlackBerry', 291
20. 'Tweetbot for Mac', 289


    \end{Verbatim}

    appUsed() method

The appUsed() method first filters the DataFrame, keeping only the
source field in each entry. Each source field (in each entry), is added
to a dictionary; if it is a new key it sets the number of occurances
(value) as 1, otherwise increasing the value by 1. The resultant
dictionary is then sorted by largest to smallest keys (number of
occurances). A for loop then iterates over each key-value pair in the
dictionary, generating a human-readable string representing the contents
of the dictionary, which is then returned and then printed.

    \begin{Verbatim}[commandchars=\\\{\}]
{\color{incolor}In [{\color{incolor}21}]:} \PY{k}{def} \PY{n+nf}{appUsed}\PY{p}{(}\PY{n+nb+bp}{self}\PY{p}{)}\PY{p}{:}
             \PY{n}{apps} \PY{o}{=} \PY{n+nb+bp}{self}\PY{o}{.}\PY{n}{data}\PY{p}{[}\PY{l+s+s1}{\PYZsq{}}\PY{l+s+s1}{source}\PY{l+s+s1}{\PYZsq{}}\PY{p}{]}\PY{o}{.}\PY{n}{str}\PY{o}{.}\PY{n}{extract}\PY{p}{(}\PY{l+s+s2}{\PYZdq{}}\PY{l+s+s2}{\PYZlt{}.*\PYZgt{}(.*)\PYZlt{}/a\PYZgt{}}\PY{l+s+s2}{\PYZdq{}}\PY{p}{,} \PY{n}{expand} \PY{o}{=} \PY{k+kc}{False}\PY{p}{)}
             \PY{n}{platforms} \PY{o}{=} \PY{p}{\PYZob{}}\PY{p}{\PYZcb{}}
             \PY{k}{for} \PY{n}{x} \PY{o+ow}{in} \PY{n}{apps}\PY{p}{:}
                 \PY{n}{y} \PY{o}{=} \PY{n+nb}{str}\PY{p}{(}\PY{n}{x}\PY{p}{)}
                 \PY{k}{if} \PY{n}{y} \PY{o+ow}{in} \PY{n}{platforms}\PY{p}{:}
                     \PY{n}{platforms}\PY{p}{[}\PY{n}{y}\PY{p}{]} \PY{o}{+}\PY{o}{=} \PY{l+m+mi}{1}
                 \PY{k}{else}\PY{p}{:}
                     \PY{n}{platforms}\PY{p}{[}\PY{n}{y}\PY{p}{]} \PY{o}{=} \PY{l+m+mi}{1}
             \PY{k+kn}{import} \PY{n+nn}{operator}
             \PY{n}{sortPlat}\PY{o}{=} \PY{n+nb}{sorted}\PY{p}{(}\PY{n}{platforms}\PY{o}{.}\PY{n}{items}\PY{p}{(}\PY{p}{)}\PY{p}{,} \PY{n}{key}\PY{o}{=}\PY{n}{operator}\PY{o}{.}\PY{n}{itemgetter}\PY{p}{(}\PY{l+m+mi}{1}\PY{p}{)}\PY{p}{,} \PY{n}{reverse}\PY{o}{=}\PY{k+kc}{True}\PY{p}{)}
             
             \PY{n}{returnString} \PY{o}{=} \PY{l+s+s2}{\PYZdq{}}\PY{l+s+s2}{Top Platforms used }\PY{l+s+se}{\PYZbs{}n}\PY{l+s+s2}{\PYZdq{}}
             \PY{k}{for} \PY{n}{m} \PY{o+ow}{in} \PY{n+nb}{range}\PY{p}{(}\PY{l+m+mi}{0}\PY{p}{,}\PY{l+m+mi}{20}\PY{p}{)}\PY{p}{:}
                 \PY{n}{returnString} \PY{o}{+}\PY{o}{=} \PY{p}{(}\PY{n+nb}{str}\PY{p}{(}\PY{n}{m} \PY{o}{+} \PY{l+m+mi}{1}\PY{p}{)} \PY{o}{+} \PY{l+s+s2}{\PYZdq{}}\PY{l+s+s2}{. }\PY{l+s+s2}{\PYZdq{}} \PY{o}{+} \PY{n+nb}{str}\PY{p}{(}\PY{n}{sortPlat}\PY{p}{[}\PY{n}{m}\PY{p}{]}\PY{p}{)} \PY{o}{+} \PY{l+s+s2}{\PYZdq{}}\PY{l+s+se}{\PYZbs{}n}\PY{l+s+s2}{\PYZdq{}}\PY{p}{)}\PY{o}{.}\PY{n}{replace}\PY{p}{(}\PY{l+s+s2}{\PYZdq{}}\PY{l+s+s2}{(}\PY{l+s+s2}{\PYZdq{}}\PY{p}{,} \PY{l+s+s2}{\PYZdq{}}\PY{l+s+s2}{\PYZdq{}}\PY{p}{)}\PY{o}{.}\PY{n}{replace}\PY{p}{(}\PY{l+s+s2}{\PYZdq{}}\PY{l+s+s2}{)}\PY{l+s+s2}{\PYZdq{}}\PY{p}{,} \PY{l+s+s2}{\PYZdq{}}\PY{l+s+s2}{\PYZdq{}}\PY{p}{)}
             \PY{k}{return} \PY{n}{returnString}
\end{Verbatim}


    Tweet Activity Timelines

    \begin{Verbatim}[commandchars=\\\{\}]
{\color{incolor}In [{\color{incolor}22}]:} \PY{k+kn}{from} \PY{n+nn}{parsing} \PY{k}{import} \PY{o}{*}
         \PY{k+kn}{from} \PY{n+nn}{plotter} \PY{k}{import} \PY{o}{*}
         
         \PY{n}{cr} \PY{o}{=} \PY{n}{CensusReader}\PY{p}{(}\PY{l+s+s2}{\PYZdq{}}\PY{l+s+s2}{data/CometLanding\PYZus{}refined.csv}\PY{l+s+s2}{\PYZdq{}}\PY{p}{)}
         \PY{n}{data} \PY{o}{=} \PY{n}{cr}\PY{o}{.}\PY{n}{data}
         
         \PY{n}{plotter} \PY{o}{=} \PY{n}{Plotter}\PY{p}{(}\PY{p}{)}
         
         \PY{n}{plotter}\PY{o}{.}\PY{n}{tweetsTimeLine}\PY{p}{(}\PY{n}{data}\PY{p}{)}
         \PY{n}{plotter}\PY{o}{.}\PY{n}{retweetsTimeLine}\PY{p}{(}\PY{n}{data}\PY{p}{)}
         \PY{n}{plotter}\PY{o}{.}\PY{n}{repliesTimeLine}\PY{p}{(}\PY{n}{data}\PY{p}{)}
\end{Verbatim}


    \begin{center}
    \adjustimage{max size={0.9\linewidth}{0.9\paperheight}}{output_44_0.png}
    \end{center}
    { \hspace*{\fill} \\}
    
    \begin{center}
    \adjustimage{max size={0.9\linewidth}{0.9\paperheight}}{output_44_1.png}
    \end{center}
    { \hspace*{\fill} \\}
    
    \begin{center}
    \adjustimage{max size={0.9\linewidth}{0.9\paperheight}}{output_44_2.png}
    \end{center}
    { \hspace*{\fill} \\}
    
    tweetsTimeLine() method

The tweetsTimeLine() method begins by filtering out any retweets,
finding only valid tweets. It then groups each entry in the DataFrame of
valid tweets by the time field, using the pandas groupby() method. Each
tweet time is then iterated over, and the number of tweets at each time
is found, and stored in a separate list, num. These lists are
preprocessed and then passed into methods imported from
matplotlib.pyplot (as plt) to draw the timeline before it is displayed.

    \begin{Verbatim}[commandchars=\\\{\}]
{\color{incolor}In [{\color{incolor}23}]:} \PY{k}{def} \PY{n+nf}{tweetsTimeLine}\PY{p}{(}\PY{n+nb+bp}{self}\PY{p}{,} \PY{n}{data}\PY{p}{)}\PY{p}{:}
             \PY{n}{tweets} \PY{o}{=} \PY{n}{data}\PY{p}{[}\PY{n}{data}\PY{p}{[}\PY{l+s+s1}{\PYZsq{}}\PY{l+s+s1}{text}\PY{l+s+s1}{\PYZsq{}}\PY{p}{]}\PY{o}{.}\PY{n}{str}\PY{o}{.}\PY{n}{startswith}\PY{p}{(}\PY{l+s+s2}{\PYZdq{}}\PY{l+s+s2}{RT}\PY{l+s+s2}{\PYZdq{}}\PY{p}{)} \PY{o}{==} \PY{k+kc}{False}\PY{p}{]}
             \PY{n}{groupbytime} \PY{o}{=} \PY{n}{tweets}\PY{o}{.}\PY{n}{groupby}\PY{p}{(}\PY{l+s+s1}{\PYZsq{}}\PY{l+s+s1}{time}\PY{l+s+s1}{\PYZsq{}}\PY{p}{)}\PY{o}{.}\PY{n}{count}\PY{p}{(}\PY{p}{)}
             \PY{n}{time} \PY{o}{=} \PY{n}{groupbytime}\PY{o}{.}\PY{n}{iloc}\PY{p}{[}\PY{p}{:}\PY{p}{,} \PY{l+m+mi}{0}\PY{p}{]}
             \PY{n}{count} \PY{o}{=} \PY{l+m+mi}{0}
             \PY{n}{num} \PY{o}{=} \PY{p}{[}\PY{p}{]} \PY{c+c1}{\PYZsh{}number of tweets}
             \PY{n}{xticks} \PY{o}{=} \PY{p}{[}\PY{p}{]}
             \PY{k}{for} \PY{n}{i}\PY{p}{,} \PY{n}{v} \PY{o+ow}{in} \PY{n}{time}\PY{o}{.}\PY{n}{iteritems}\PY{p}{(}\PY{p}{)}\PY{p}{:}
                 \PY{n}{count} \PY{o}{+}\PY{o}{=} \PY{l+m+mi}{1}
                 \PY{n}{num}\PY{o}{.}\PY{n}{append}\PY{p}{(}\PY{n}{v}\PY{p}{)}
                 \PY{n}{xticks}\PY{o}{.}\PY{n}{append}\PY{p}{(}\PY{n}{i}\PY{p}{)}
             
             \PY{n}{y} \PY{o}{=} \PY{n}{num}
             \PY{n}{fig}\PY{p}{,} \PY{n}{ax} \PY{o}{=} \PY{n}{plt}\PY{o}{.}\PY{n}{subplots}\PY{p}{(}\PY{p}{)}
             \PY{n}{ax}\PY{o}{.}\PY{n}{plot}\PY{p}{(}\PY{n}{y}\PY{p}{)}
             \PY{n}{start}\PY{p}{,} \PY{n}{end} \PY{o}{=} \PY{n}{ax}\PY{o}{.}\PY{n}{get\PYZus{}xlim}\PY{p}{(}\PY{p}{)}
             \PY{n}{ax}\PY{o}{.}\PY{n}{xaxis}\PY{o}{.}\PY{n}{set\PYZus{}ticks}\PY{p}{(}\PY{n}{np}\PY{o}{.}\PY{n}{arange}\PY{p}{(}\PY{n}{start}\PY{p}{,} \PY{n}{end}\PY{p}{,} \PY{l+m+mi}{300}\PY{p}{)}\PY{p}{)}
             \PY{n}{ax}\PY{o}{.}\PY{n}{xaxis}\PY{o}{.}\PY{n}{set\PYZus{}major\PYZus{}formatter}\PY{p}{(}\PY{n}{ticker}\PY{o}{.}\PY{n}{FormatStrFormatter}\PY{p}{(}\PY{l+s+s1}{\PYZsq{}}\PY{l+s+si}{\PYZpc{}0.1f}\PY{l+s+s1}{\PYZsq{}}\PY{p}{)}\PY{p}{)}
             \PY{n}{ax}\PY{o}{.}\PY{n}{set\PYZus{}xticklabels}\PY{p}{(}\PY{n}{xticks}\PY{p}{,} \PY{n}{rotation}\PY{o}{=}\PY{l+s+s1}{\PYZsq{}}\PY{l+s+s1}{vertical}\PY{l+s+s1}{\PYZsq{}}\PY{p}{,} \PY{n}{fontsize}\PY{o}{=}\PY{l+m+mi}{8}\PY{p}{)}
             \PY{n}{ax}\PY{o}{.}\PY{n}{set\PYZus{}xlabel}\PY{p}{(}\PY{l+s+s1}{\PYZsq{}}\PY{l+s+s1}{Time and Date}\PY{l+s+s1}{\PYZsq{}}\PY{p}{)}
             \PY{n}{ax}\PY{o}{.}\PY{n}{set\PYZus{}ylabel}\PY{p}{(}\PY{l+s+s1}{\PYZsq{}}\PY{l+s+s1}{Number of Tweets}\PY{l+s+s1}{\PYZsq{}}\PY{p}{)}
             \PY{n}{plt}\PY{o}{.}\PY{n}{show}\PY{p}{(}\PY{p}{)}
\end{Verbatim}


    retweetsTimeLine() method

The retweetsTimeLine method is largely identical to tweetsTimeLine().
The differences between the two are axes' labels, and that the tweets
used in this case are filtered to be retweets. This is done by ensuring
the text field of each entry in the DataFrame (representing the tweet
text) starts with "RT".

    \begin{Verbatim}[commandchars=\\\{\}]
{\color{incolor}In [{\color{incolor}24}]:} \PY{k}{def} \PY{n+nf}{retweetsTimeLine}\PY{p}{(}\PY{n+nb+bp}{self}\PY{p}{,} \PY{n}{data}\PY{p}{)}\PY{p}{:}
             \PY{n}{retweets} \PY{o}{=} \PY{n}{data}\PY{p}{[}\PY{n}{data}\PY{p}{[}\PY{l+s+s1}{\PYZsq{}}\PY{l+s+s1}{text}\PY{l+s+s1}{\PYZsq{}}\PY{p}{]}\PY{o}{.}\PY{n}{str}\PY{o}{.}\PY{n}{startswith}\PY{p}{(}\PY{l+s+s2}{\PYZdq{}}\PY{l+s+s2}{RT}\PY{l+s+s2}{\PYZdq{}}\PY{p}{)} \PY{o}{==} \PY{k+kc}{True}\PY{p}{]}
             \PY{n}{groupbytime} \PY{o}{=} \PY{n}{retweets}\PY{o}{.}\PY{n}{groupby}\PY{p}{(}\PY{l+s+s1}{\PYZsq{}}\PY{l+s+s1}{time}\PY{l+s+s1}{\PYZsq{}}\PY{p}{)}\PY{o}{.}\PY{n}{count}\PY{p}{(}\PY{p}{)}
             \PY{n}{time} \PY{o}{=} \PY{n}{groupbytime}\PY{o}{.}\PY{n}{iloc}\PY{p}{[}\PY{p}{:}\PY{p}{,} \PY{l+m+mi}{0}\PY{p}{]}
             \PY{n}{count} \PY{o}{=} \PY{l+m+mi}{0}
             \PY{n}{num} \PY{o}{=} \PY{p}{[}\PY{p}{]} \PY{c+c1}{\PYZsh{}number of tweets}
             \PY{n}{xticks} \PY{o}{=} \PY{p}{[}\PY{p}{]}
             \PY{k}{for} \PY{n}{i}\PY{p}{,} \PY{n}{v} \PY{o+ow}{in} \PY{n}{time}\PY{o}{.}\PY{n}{iteritems}\PY{p}{(}\PY{p}{)}\PY{p}{:}
                 \PY{n}{count} \PY{o}{+}\PY{o}{=} \PY{l+m+mi}{1}
                 \PY{n}{num}\PY{o}{.}\PY{n}{append}\PY{p}{(}\PY{n}{v}\PY{p}{)}
                 \PY{n}{xticks}\PY{o}{.}\PY{n}{append}\PY{p}{(}\PY{n}{i}\PY{p}{)}
             
             \PY{n}{y} \PY{o}{=} \PY{n}{num}
             \PY{n}{fig}\PY{p}{,} \PY{n}{ax} \PY{o}{=} \PY{n}{plt}\PY{o}{.}\PY{n}{subplots}\PY{p}{(}\PY{p}{)}
             \PY{n}{ax}\PY{o}{.}\PY{n}{plot}\PY{p}{(}\PY{n}{y}\PY{p}{)}
             \PY{n}{start}\PY{p}{,} \PY{n}{end} \PY{o}{=} \PY{n}{ax}\PY{o}{.}\PY{n}{get\PYZus{}xlim}\PY{p}{(}\PY{p}{)}
             \PY{n}{ax}\PY{o}{.}\PY{n}{xaxis}\PY{o}{.}\PY{n}{set\PYZus{}ticks}\PY{p}{(}\PY{n}{np}\PY{o}{.}\PY{n}{arange}\PY{p}{(}\PY{n}{start}\PY{p}{,} \PY{n}{end}\PY{p}{,} \PY{l+m+mi}{300}\PY{p}{)}\PY{p}{)}
             \PY{n}{ax}\PY{o}{.}\PY{n}{xaxis}\PY{o}{.}\PY{n}{set\PYZus{}major\PYZus{}formatter}\PY{p}{(}\PY{n}{ticker}\PY{o}{.}\PY{n}{FormatStrFormatter}\PY{p}{(}\PY{l+s+s1}{\PYZsq{}}\PY{l+s+si}{\PYZpc{}0.1f}\PY{l+s+s1}{\PYZsq{}}\PY{p}{)}\PY{p}{)}
             \PY{n}{ax}\PY{o}{.}\PY{n}{set\PYZus{}xticklabels}\PY{p}{(}\PY{n}{xticks}\PY{p}{,} \PY{n}{rotation}\PY{o}{=}\PY{l+s+s1}{\PYZsq{}}\PY{l+s+s1}{vertical}\PY{l+s+s1}{\PYZsq{}}\PY{p}{,} \PY{n}{fontsize}\PY{o}{=}\PY{l+m+mi}{8}\PY{p}{)}
             \PY{n}{ax}\PY{o}{.}\PY{n}{set\PYZus{}xlabel}\PY{p}{(}\PY{l+s+s1}{\PYZsq{}}\PY{l+s+s1}{Time and Date}\PY{l+s+s1}{\PYZsq{}}\PY{p}{)}
             \PY{n}{ax}\PY{o}{.}\PY{n}{set\PYZus{}ylabel}\PY{p}{(}\PY{l+s+s1}{\PYZsq{}}\PY{l+s+s1}{Number of Retweets}\PY{l+s+s1}{\PYZsq{}}\PY{p}{)}
             \PY{n}{plt}\PY{o}{.}\PY{n}{show}\PY{p}{(}\PY{p}{)}
\end{Verbatim}


    repliesTimeLine() method

The repliesTimeLine method is also largely identical to
tweetsTimeLine(). The differences between the two are axes' labels, and
that the tweets used in this case are filtered to be replies. This is
done by ensuring the text 'in\_reply\_to\_screen\_name' of each entry in
the DataFrame is not null (i.e. there is someone the tweet is replying
to).

    \begin{Verbatim}[commandchars=\\\{\}]
{\color{incolor}In [{\color{incolor}25}]:} \PY{k}{def} \PY{n+nf}{repliesTimeLine}\PY{p}{(}\PY{n+nb+bp}{self}\PY{p}{,} \PY{n}{data}\PY{p}{)}\PY{p}{:}
             \PY{n}{replies} \PY{o}{=} \PY{n}{data}\PY{p}{[}\PY{n}{data}\PY{p}{[}\PY{l+s+s1}{\PYZsq{}}\PY{l+s+s1}{in\PYZus{}reply\PYZus{}to\PYZus{}screen\PYZus{}name}\PY{l+s+s1}{\PYZsq{}}\PY{p}{]}\PY{o}{.}\PY{n}{notnull}\PY{p}{(}\PY{p}{)} \PY{o}{==} \PY{k+kc}{True}\PY{p}{]}
             \PY{n}{groupbytime} \PY{o}{=} \PY{n}{replies}\PY{o}{.}\PY{n}{groupby}\PY{p}{(}\PY{l+s+s1}{\PYZsq{}}\PY{l+s+s1}{time}\PY{l+s+s1}{\PYZsq{}}\PY{p}{)}\PY{o}{.}\PY{n}{count}\PY{p}{(}\PY{p}{)}
             \PY{n}{time} \PY{o}{=} \PY{n}{groupbytime}\PY{o}{.}\PY{n}{iloc}\PY{p}{[}\PY{p}{:}\PY{p}{,} \PY{l+m+mi}{0}\PY{p}{]}
             \PY{n}{count} \PY{o}{=} \PY{l+m+mi}{0}
             \PY{n}{num} \PY{o}{=} \PY{p}{[}\PY{p}{]} \PY{c+c1}{\PYZsh{}number of tweets}
             \PY{n}{xticks} \PY{o}{=} \PY{p}{[}\PY{p}{]}
             \PY{k}{for} \PY{n}{i}\PY{p}{,} \PY{n}{v} \PY{o+ow}{in} \PY{n}{time}\PY{o}{.}\PY{n}{iteritems}\PY{p}{(}\PY{p}{)}\PY{p}{:}
                 \PY{n}{count} \PY{o}{+}\PY{o}{=} \PY{l+m+mi}{1}
                 \PY{n}{num}\PY{o}{.}\PY{n}{append}\PY{p}{(}\PY{n}{v}\PY{p}{)}
                 \PY{n}{xticks}\PY{o}{.}\PY{n}{append}\PY{p}{(}\PY{n}{i}\PY{p}{)}
             
             \PY{n}{y} \PY{o}{=} \PY{n}{num}
             \PY{n}{fig}\PY{p}{,} \PY{n}{ax} \PY{o}{=} \PY{n}{plt}\PY{o}{.}\PY{n}{subplots}\PY{p}{(}\PY{p}{)}
             \PY{n}{ax}\PY{o}{.}\PY{n}{plot}\PY{p}{(}\PY{n}{y}\PY{p}{)}
             \PY{n}{start}\PY{p}{,} \PY{n}{end} \PY{o}{=} \PY{n}{ax}\PY{o}{.}\PY{n}{get\PYZus{}xlim}\PY{p}{(}\PY{p}{)}
             \PY{n}{ax}\PY{o}{.}\PY{n}{xaxis}\PY{o}{.}\PY{n}{set\PYZus{}ticks}\PY{p}{(}\PY{n}{np}\PY{o}{.}\PY{n}{arange}\PY{p}{(}\PY{n}{start}\PY{p}{,} \PY{n}{end}\PY{p}{,} \PY{l+m+mi}{300}\PY{p}{)}\PY{p}{)}
             \PY{n}{ax}\PY{o}{.}\PY{n}{xaxis}\PY{o}{.}\PY{n}{set\PYZus{}major\PYZus{}formatter}\PY{p}{(}\PY{n}{ticker}\PY{o}{.}\PY{n}{FormatStrFormatter}\PY{p}{(}\PY{l+s+s1}{\PYZsq{}}\PY{l+s+si}{\PYZpc{}0.1f}\PY{l+s+s1}{\PYZsq{}}\PY{p}{)}\PY{p}{)}
             \PY{n}{ax}\PY{o}{.}\PY{n}{set\PYZus{}xticklabels}\PY{p}{(}\PY{n}{xticks}\PY{p}{,} \PY{n}{rotation}\PY{o}{=}\PY{l+s+s1}{\PYZsq{}}\PY{l+s+s1}{vertical}\PY{l+s+s1}{\PYZsq{}}\PY{p}{,} \PY{n}{fontsize}\PY{o}{=}\PY{l+m+mi}{8}\PY{p}{)}
             \PY{n}{ax}\PY{o}{.}\PY{n}{set\PYZus{}xlabel}\PY{p}{(}\PY{l+s+s1}{\PYZsq{}}\PY{l+s+s1}{Time and Date}\PY{l+s+s1}{\PYZsq{}}\PY{p}{)}
             \PY{n}{ax}\PY{o}{.}\PY{n}{set\PYZus{}ylabel}\PY{p}{(}\PY{l+s+s1}{\PYZsq{}}\PY{l+s+s1}{Number of Replies}\PY{l+s+s1}{\PYZsq{}}\PY{p}{)}
             \PY{n}{plt}\PY{o}{.}\PY{n}{show}\PY{p}{(}\PY{p}{)}
\end{Verbatim}


    Finding and analysing user interactions

    Finding and analysing user interactions

The initial list of replies, retweets, and mentions are found by
selecting from the DataFrame. The replies are found by only keeping
entities whose value for the '' field is not null; retweets are found by
checking the 'text' field of an entity starts with "RT"; mentions are
found by checking that the 'text' field of an entity contains an @
symbol, and also that the requirements for replies and retweets both
fail. This ensures retweets and replies are not counted as mentions. The
retweetsToTwoUsers(), repliesToTwoUsers(), and mentionsToTwoUsers()
methods are then used to convert the resultant variables retweets,
replies, and mentions into 2D arrays of the usernames of both users
involved in the interaction. A resultant list of all interactions is
formed by concatenating all 3 formed arrays; the lengths of each list of
interactions at each stage of processing are then analysed.

    \begin{Verbatim}[commandchars=\\\{\}]
{\color{incolor}In [{\color{incolor}26}]:} \PY{k+kn}{from} \PY{n+nn}{parsing} \PY{k}{import} \PY{o}{*}
         \PY{k+kn}{from} \PY{n+nn}{plotter} \PY{k}{import} \PY{o}{*}
         
         \PY{n}{plotter} \PY{o}{=} \PY{n}{Plotter}\PY{p}{(}\PY{p}{)}
         
         \PY{n+nb}{print}\PY{p}{(}\PY{l+s+s2}{\PYZdq{}}\PY{l+s+s2}{Finding user interactions.}\PY{l+s+se}{\PYZbs{}n}\PY{l+s+s2}{\PYZdq{}}\PY{p}{)}
         
         \PY{n}{data} \PY{o}{=} \PY{n}{CensusReader}\PY{p}{(}\PY{l+s+s2}{\PYZdq{}}\PY{l+s+s2}{data/CometLanding\PYZus{}refined.csv}\PY{l+s+s2}{\PYZdq{}}\PY{p}{)}\PY{o}{.}\PY{n}{data}
         
         \PY{n}{replies} \PY{o}{=} \PY{n}{data}\PY{p}{[}\PY{n}{data}\PY{p}{[}\PY{l+s+s1}{\PYZsq{}}\PY{l+s+s1}{in\PYZus{}reply\PYZus{}to\PYZus{}screen\PYZus{}name}\PY{l+s+s1}{\PYZsq{}}\PY{p}{]}\PY{o}{.}\PY{n}{notnull}\PY{p}{(}\PY{p}{)}\PY{p}{]}
         \PY{n}{retweets} \PY{o}{=} \PY{n}{data}\PY{p}{[}\PY{n}{data}\PY{p}{[}\PY{l+s+s1}{\PYZsq{}}\PY{l+s+s1}{text}\PY{l+s+s1}{\PYZsq{}}\PY{p}{]}\PY{o}{.}\PY{n}{str}\PY{o}{.}\PY{n}{startswith}\PY{p}{(}\PY{l+s+s2}{\PYZdq{}}\PY{l+s+s2}{RT}\PY{l+s+s2}{\PYZdq{}}\PY{p}{)} \PY{o}{==} \PY{k+kc}{True}\PY{p}{]}
         
         \PY{n}{mentions} \PY{o}{=} \PY{n}{data}\PY{p}{[}\PY{n}{data}\PY{p}{[}\PY{l+s+s1}{\PYZsq{}}\PY{l+s+s1}{text}\PY{l+s+s1}{\PYZsq{}}\PY{p}{]}\PY{o}{.}\PY{n}{str}\PY{o}{.}\PY{n}{contains}\PY{p}{(}\PY{l+s+s2}{\PYZdq{}}\PY{l+s+s2}{@}\PY{l+s+s2}{\PYZdq{}}\PY{p}{)} \PY{o}{==} \PY{k+kc}{True}\PY{p}{]}
         \PY{n}{mentions} \PY{o}{=}  \PY{n}{mentions}\PY{p}{[}\PY{n}{mentions}\PY{p}{[}\PY{l+s+s1}{\PYZsq{}}\PY{l+s+s1}{in\PYZus{}reply\PYZus{}to\PYZus{}screen\PYZus{}name}\PY{l+s+s1}{\PYZsq{}}\PY{p}{]}\PY{o}{.}\PY{n}{notnull}\PY{p}{(}\PY{p}{)} \PY{o}{==} \PY{k+kc}{False}\PY{p}{]}
         \PY{n}{mentions} \PY{o}{=} \PY{n}{mentions}\PY{p}{[}\PY{n}{mentions}\PY{p}{[}\PY{l+s+s1}{\PYZsq{}}\PY{l+s+s1}{text}\PY{l+s+s1}{\PYZsq{}}\PY{p}{]}\PY{o}{.}\PY{n}{str}\PY{o}{.}\PY{n}{startswith}\PY{p}{(}\PY{l+s+s2}{\PYZdq{}}\PY{l+s+s2}{RT}\PY{l+s+s2}{\PYZdq{}}\PY{p}{)} \PY{o}{==} \PY{k+kc}{False}\PY{p}{]}
         
         \PY{n+nb}{print}\PY{p}{(}\PY{l+s+s2}{\PYZdq{}}\PY{l+s+s2}{There were }\PY{l+s+s2}{\PYZdq{}} \PY{o}{+} \PY{n+nb}{str}\PY{p}{(}\PY{n+nb}{len}\PY{p}{(}\PY{n}{replies}\PY{p}{)}\PY{p}{)} \PY{o}{+} \PY{l+s+s2}{\PYZdq{}}\PY{l+s+s2}{ replies prior to processing.}\PY{l+s+s2}{\PYZdq{}}\PY{p}{)}
         \PY{n+nb}{print}\PY{p}{(}\PY{l+s+s2}{\PYZdq{}}\PY{l+s+s2}{There were }\PY{l+s+s2}{\PYZdq{}} \PY{o}{+} \PY{n+nb}{str}\PY{p}{(}\PY{n+nb}{len}\PY{p}{(}\PY{n}{retweets}\PY{p}{)}\PY{p}{)} \PY{o}{+} \PY{l+s+s2}{\PYZdq{}}\PY{l+s+s2}{ retweets prior to processing.}\PY{l+s+s2}{\PYZdq{}}\PY{p}{)}
         \PY{n+nb}{print}\PY{p}{(}\PY{l+s+s2}{\PYZdq{}}\PY{l+s+s2}{There were }\PY{l+s+s2}{\PYZdq{}} \PY{o}{+} \PY{n+nb}{str}\PY{p}{(}\PY{n+nb}{len}\PY{p}{(}\PY{n}{mentions}\PY{p}{)}\PY{p}{)} \PY{o}{+} \PY{l+s+s2}{\PYZdq{}}\PY{l+s+s2}{ mentions containing mentions prior to processing.}\PY{l+s+se}{\PYZbs{}n}\PY{l+s+s2}{\PYZdq{}}\PY{p}{)}
         
         \PY{n}{formattedRetweets} \PY{o}{=} \PY{n}{plotter}\PY{o}{.}\PY{n}{retweetsToTwoUsers}\PY{p}{(}\PY{n}{retweets}\PY{p}{)}
         \PY{n}{formattedReplies} \PY{o}{=} \PY{n}{plotter}\PY{o}{.}\PY{n}{repliesToTwoUsers}\PY{p}{(}\PY{n}{replies}\PY{p}{)}
         \PY{n}{formattedMentions} \PY{o}{=} \PY{n}{plotter}\PY{o}{.}\PY{n}{mentionsToTwoUsers}\PY{p}{(}\PY{n}{mentions}\PY{p}{)}
         
         \PY{n}{interactions} \PY{o}{=} \PY{n}{np}\PY{o}{.}\PY{n}{concatenate}\PY{p}{(}\PY{p}{(}\PY{n}{formattedReplies}\PY{p}{,} \PY{n}{formattedMentions}\PY{p}{,} \PY{n}{formattedRetweets}\PY{p}{)}\PY{p}{,} \PY{n}{axis}\PY{o}{=}\PY{l+m+mi}{0}\PY{p}{)}
         
         \PY{n}{retweets} \PY{o}{=} \PY{n+nb}{len}\PY{p}{(}\PY{n}{formattedRetweets}\PY{p}{)}
         \PY{n}{replies} \PY{o}{=} \PY{n+nb}{len}\PY{p}{(}\PY{n}{formattedReplies}\PY{p}{)}
         \PY{n}{mentions} \PY{o}{=} \PY{n+nb}{len}\PY{p}{(}\PY{n}{formattedMentions}\PY{p}{)}
         
         \PY{n+nb}{print}\PY{p}{(}\PY{l+s+s2}{\PYZdq{}}\PY{l+s+s2}{There were }\PY{l+s+s2}{\PYZdq{}} \PY{o}{+} \PY{n+nb}{str}\PY{p}{(}\PY{n}{replies}\PY{p}{)} \PY{o}{+} \PY{l+s+s2}{\PYZdq{}}\PY{l+s+s2}{ replies after processing.}\PY{l+s+s2}{\PYZdq{}}\PY{p}{)}
         \PY{n+nb}{print}\PY{p}{(}\PY{l+s+s2}{\PYZdq{}}\PY{l+s+s2}{There were }\PY{l+s+s2}{\PYZdq{}} \PY{o}{+} \PY{n+nb}{str}\PY{p}{(}\PY{n}{retweets}\PY{p}{)} \PY{o}{+} \PY{l+s+s2}{\PYZdq{}}\PY{l+s+s2}{ retweets after processing.}\PY{l+s+s2}{\PYZdq{}}\PY{p}{)}
         \PY{n+nb}{print}\PY{p}{(}\PY{l+s+s2}{\PYZdq{}}\PY{l+s+s2}{There were }\PY{l+s+s2}{\PYZdq{}} \PY{o}{+} \PY{n+nb}{str}\PY{p}{(}\PY{n}{mentions}\PY{p}{)} \PY{o}{+} \PY{l+s+s2}{\PYZdq{}}\PY{l+s+s2}{ mentions after processing.}\PY{l+s+s2}{\PYZdq{}}\PY{p}{)}
         \PY{n+nb}{print}\PY{p}{(}\PY{l+s+s2}{\PYZdq{}}\PY{l+s+s2}{Note: a tweet containing \PYZgt{}0 mentions can contain \PYZgt{}1 mention.}\PY{l+s+se}{\PYZbs{}n}\PY{l+s+s2}{\PYZdq{}}\PY{p}{)}
         
         \PY{n+nb}{print}\PY{p}{(}\PY{l+s+s2}{\PYZdq{}}\PY{l+s+s2}{There were }\PY{l+s+s2}{\PYZdq{}} \PY{o}{+} \PY{n+nb}{str}\PY{p}{(}\PY{n+nb}{len}\PY{p}{(}\PY{n}{interactions}\PY{p}{)}\PY{p}{)} \PY{o}{+} \PY{l+s+s2}{\PYZdq{}}\PY{l+s+s2}{ interactions in total after processing.}\PY{l+s+se}{\PYZbs{}n}\PY{l+s+se}{\PYZbs{}n}\PY{l+s+s2}{\PYZdq{}}\PY{p}{)}
\end{Verbatim}


    \begin{Verbatim}[commandchars=\\\{\}]
Finding user interactions.

There were 1723 replies prior to processing.
There were 59998 retweets prior to processing.
There were 5044 mentions containing mentions prior to processing.

There were 1641 replies after processing.
There were 53377 retweets after processing.
There were 5972 mentions after processing.
Note: a tweet containing >0 mentions can contain >1 mention.

There were 60990 interactions in total after processing.



    \end{Verbatim}

    retweetsToTwoUsers() method

The retweetsToTwoUsers() method takes in a DataFrame containing
retweets, iterating over each entry (tweet). The user who sent the tweet
is found from the 'from\_user' field. The user being retweeted is found
by parsing the tweet text, in the 'text' field. Both usernames are then
added to the list of interactions, as a sub-array. The list of
interactions is returned after duplicates are removed from it.

    \begin{Verbatim}[commandchars=\\\{\}]
{\color{incolor}In [{\color{incolor}27}]:} \PY{k}{def} \PY{n+nf}{retweetsToTwoUsers}\PY{p}{(}\PY{n+nb+bp}{self}\PY{p}{,} \PY{n}{data}\PY{p}{)}\PY{p}{:}
             \PY{n}{interactions} \PY{o}{=} \PY{p}{[}\PY{p}{]}
             \PY{k}{for} \PY{n}{index}\PY{p}{,} \PY{n}{row} \PY{o+ow}{in} \PY{n}{data}\PY{o}{.}\PY{n}{iterrows}\PY{p}{(}\PY{p}{)}\PY{p}{:}
                 \PY{n}{sentBy} \PY{o}{=} \PY{n}{row}\PY{p}{[}\PY{l+s+s1}{\PYZsq{}}\PY{l+s+s1}{from\PYZus{}user}\PY{l+s+s1}{\PYZsq{}}\PY{p}{]}
                 \PY{n}{tweetText} \PY{o}{=} \PY{n}{row}\PY{p}{[}\PY{l+s+s1}{\PYZsq{}}\PY{l+s+s1}{text}\PY{l+s+s1}{\PYZsq{}}\PY{p}{]}
                 
                 \PY{k}{if} \PY{n}{sentBy} \PY{o}{==} \PY{k+kc}{None} \PY{o+ow}{or} \PY{n}{tweetText} \PY{o}{==} \PY{k+kc}{None} \PY{o+ow}{or} \PY{n+nb}{type}\PY{p}{(}\PY{n}{tweetText}\PY{p}{)} \PY{o}{!=} \PY{n+nb}{str} \PY{o+ow}{or} \PY{n+nb}{type}\PY{p}{(}\PY{n}{sentBy}\PY{p}{)} \PY{o}{!=} \PY{n+nb}{str}\PY{p}{:} \PY{k}{continue}
                 \PY{n}{splitText} \PY{o}{=} \PY{n}{tweetText}\PY{o}{.}\PY{n}{split}\PY{p}{(}\PY{p}{)}
                 \PY{k}{if} \PY{n+nb}{len}\PY{p}{(}\PY{n}{splitText}\PY{p}{)} \PY{o}{\PYZlt{}} \PY{l+m+mi}{2}\PY{p}{:} \PY{k}{continue}
                 
                 \PY{n}{retweeting} \PY{o}{=} \PY{n}{splitText}\PY{p}{[}\PY{l+m+mi}{1}\PY{p}{]}\PY{o}{.}\PY{n}{replace}\PY{p}{(}\PY{l+s+s2}{\PYZdq{}}\PY{l+s+s2}{:}\PY{l+s+s2}{\PYZdq{}}\PY{p}{,} \PY{l+s+s2}{\PYZdq{}}\PY{l+s+s2}{\PYZdq{}}\PY{p}{)}\PY{o}{.}\PY{n}{replace}\PY{p}{(}\PY{l+s+s2}{\PYZdq{}}\PY{l+s+s2}{@}\PY{l+s+s2}{\PYZdq{}}\PY{p}{,} \PY{l+s+s2}{\PYZdq{}}\PY{l+s+s2}{\PYZdq{}}\PY{p}{)}
                 \PY{n}{interactions}\PY{o}{.}\PY{n}{append}\PY{p}{(}\PY{p}{[}\PY{n}{sentBy}\PY{p}{,} \PY{n}{retweeting}\PY{p}{]}\PY{p}{)}
             
             \PY{k}{return} \PY{n}{DataFrame}\PY{p}{(}\PY{n}{interactions}\PY{p}{)}\PY{o}{.}\PY{n}{drop\PYZus{}duplicates}\PY{p}{(}\PY{p}{)}
\end{Verbatim}


    repliesToTwoUsers() method

The repliesToTwoUsers() method is largely the same as
retweetsToTwoUsers(). The second user is now the user being replied to,
and is found from the 'in\_reply\_to\_screen\_name' field.

    \begin{Verbatim}[commandchars=\\\{\}]
{\color{incolor}In [{\color{incolor}28}]:} \PY{k}{def} \PY{n+nf}{repliesToTwoUsers}\PY{p}{(}\PY{n+nb+bp}{self}\PY{p}{,} \PY{n}{data}\PY{p}{)}\PY{p}{:}
             \PY{n}{interactions} \PY{o}{=} \PY{p}{[}\PY{p}{]}
             \PY{k}{for} \PY{n}{index}\PY{p}{,} \PY{n}{row} \PY{o+ow}{in} \PY{n}{data}\PY{o}{.}\PY{n}{iterrows}\PY{p}{(}\PY{p}{)}\PY{p}{:}
                 \PY{n}{sentBy} \PY{o}{=} \PY{n}{row}\PY{p}{[}\PY{l+s+s1}{\PYZsq{}}\PY{l+s+s1}{from\PYZus{}user}\PY{l+s+s1}{\PYZsq{}}\PY{p}{]}
                 \PY{n}{inReplyTo} \PY{o}{=} \PY{n}{row}\PY{p}{[}\PY{l+s+s1}{\PYZsq{}}\PY{l+s+s1}{in\PYZus{}reply\PYZus{}to\PYZus{}screen\PYZus{}name}\PY{l+s+s1}{\PYZsq{}}\PY{p}{]}
                 \PY{k}{if} \PY{n}{sentBy} \PY{o}{==} \PY{k+kc}{None} \PY{o+ow}{or} \PY{n}{inReplyTo} \PY{o}{==} \PY{k+kc}{None} \PY{o+ow}{or} \PY{n+nb}{type}\PY{p}{(}\PY{n}{sentBy}\PY{p}{)} \PY{o}{!=} \PY{n+nb}{str} \PY{o+ow}{or} \PY{n+nb}{type}\PY{p}{(}\PY{n}{inReplyTo}\PY{p}{)} \PY{o}{!=} \PY{n+nb}{str}\PY{p}{:} \PY{k}{continue}
                 \PY{n}{interactions}\PY{o}{.}\PY{n}{append}\PY{p}{(}\PY{p}{[}\PY{n}{sentBy}\PY{p}{,} \PY{n}{inReplyTo}\PY{p}{]}\PY{p}{)}
             \PY{k}{return} \PY{n}{DataFrame}\PY{p}{(}\PY{n}{interactions}\PY{p}{)}\PY{o}{.}\PY{n}{drop\PYZus{}duplicates}\PY{p}{(}\PY{p}{)}
\end{Verbatim}


    mentionsToTwoUsers() method

The mentionsToTwoUsers() method is largely the same as
retweetsToTwoUsers(). There are now potentially more than one second
users for each first user, as a tweet sent by a single user can mention
multiple other users. The twitter text, from the 'text' field, is
therefore parsed for @ symbols; the username is extracted as the text
after each @ symbol and before the next whitespace character. Each
username mentioned is then added to the list of interactions alongside
the user who sent the tweet.

    \begin{Verbatim}[commandchars=\\\{\}]
{\color{incolor}In [{\color{incolor}29}]:} \PY{k}{def} \PY{n+nf}{mentionsToTwoUsers}\PY{p}{(}\PY{n+nb+bp}{self}\PY{p}{,} \PY{n}{data}\PY{p}{)}\PY{p}{:}
             \PY{n}{interactions} \PY{o}{=} \PY{p}{[}\PY{p}{]}
             \PY{k}{for} \PY{n}{index}\PY{p}{,} \PY{n}{row} \PY{o+ow}{in} \PY{n}{data}\PY{o}{.}\PY{n}{iterrows}\PY{p}{(}\PY{p}{)}\PY{p}{:}
                 \PY{n}{sentBy} \PY{o}{=} \PY{n}{row}\PY{p}{[}\PY{l+s+s1}{\PYZsq{}}\PY{l+s+s1}{from\PYZus{}user}\PY{l+s+s1}{\PYZsq{}}\PY{p}{]}
                 \PY{n}{tweetText} \PY{o}{=} \PY{n}{row}\PY{p}{[}\PY{l+s+s1}{\PYZsq{}}\PY{l+s+s1}{text}\PY{l+s+s1}{\PYZsq{}}\PY{p}{]}
                 
                 \PY{k}{if} \PY{n}{sentBy} \PY{o}{==} \PY{k+kc}{None} \PY{o+ow}{or} \PY{n}{tweetText} \PY{o}{==} \PY{k+kc}{None} \PY{o+ow}{or} \PY{n+nb}{type}\PY{p}{(}\PY{n}{tweetText}\PY{p}{)} \PY{o}{!=} \PY{n+nb}{str} \PY{o+ow}{or} \PY{n+nb}{type}\PY{p}{(}\PY{n}{sentBy}\PY{p}{)} \PY{o}{!=} \PY{n+nb}{str}\PY{p}{:} \PY{k}{continue}
                 \PY{n}{splitText} \PY{o}{=} \PY{n}{tweetText}\PY{o}{.}\PY{n}{split}\PY{p}{(}\PY{p}{)}
                 \PY{k}{if} \PY{n+nb}{len}\PY{p}{(}\PY{n}{splitText}\PY{p}{)} \PY{o}{\PYZlt{}} \PY{l+m+mi}{1}\PY{p}{:} \PY{k}{continue}
                 \PY{k}{else}\PY{p}{:}
                     \PY{k}{for} \PY{n}{string} \PY{o+ow}{in} \PY{n}{splitText}\PY{p}{:}
                         \PY{k}{if} \PY{p}{(}\PY{n}{string}\PY{o}{.}\PY{n}{startswith}\PY{p}{(}\PY{l+s+s2}{\PYZdq{}}\PY{l+s+s2}{@}\PY{l+s+s2}{\PYZdq{}}\PY{p}{)}\PY{p}{)}\PY{p}{:}
                             \PY{n}{interactions}\PY{o}{.}\PY{n}{append}\PY{p}{(}\PY{p}{[}\PY{n}{sentBy}\PY{p}{,} \PY{n}{string}\PY{p}{]}\PY{p}{)}
         
             \PY{k}{return} \PY{n}{DataFrame}\PY{p}{(}\PY{n}{interactions}\PY{p}{)}\PY{o}{.}\PY{n}{drop\PYZus{}duplicates}\PY{p}{(}\PY{p}{)}
\end{Verbatim}


    Network graph of all interactions

    \begin{Verbatim}[commandchars=\\\{\}]
{\color{incolor}In [{\color{incolor}30}]:} \PY{k+kn}{from} \PY{n+nn}{plotter} \PY{k}{import} \PY{o}{*}
         \PY{k+kn}{from} \PY{n+nn}{parsing} \PY{k}{import} \PY{o}{*}
         
         \PY{n}{plotter} \PY{o}{=} \PY{n}{Plotter}\PY{p}{(}\PY{p}{)}
         
         \PY{n}{data} \PY{o}{=} \PY{n}{CensusReader}\PY{p}{(}\PY{l+s+s2}{\PYZdq{}}\PY{l+s+s2}{data/CometLanding\PYZus{}refined.csv}\PY{l+s+s2}{\PYZdq{}}\PY{p}{)}\PY{o}{.}\PY{n}{data}
         
         \PY{n}{replies} \PY{o}{=} \PY{n}{data}\PY{p}{[}\PY{n}{data}\PY{p}{[}\PY{l+s+s1}{\PYZsq{}}\PY{l+s+s1}{in\PYZus{}reply\PYZus{}to\PYZus{}screen\PYZus{}name}\PY{l+s+s1}{\PYZsq{}}\PY{p}{]}\PY{o}{.}\PY{n}{notnull}\PY{p}{(}\PY{p}{)}\PY{p}{]}
         \PY{n}{retweets} \PY{o}{=} \PY{n}{data}\PY{p}{[}\PY{n}{data}\PY{p}{[}\PY{l+s+s1}{\PYZsq{}}\PY{l+s+s1}{text}\PY{l+s+s1}{\PYZsq{}}\PY{p}{]}\PY{o}{.}\PY{n}{str}\PY{o}{.}\PY{n}{startswith}\PY{p}{(}\PY{l+s+s2}{\PYZdq{}}\PY{l+s+s2}{RT}\PY{l+s+s2}{\PYZdq{}}\PY{p}{)} \PY{o}{==} \PY{k+kc}{True}\PY{p}{]}
         
         \PY{n}{mentions} \PY{o}{=} \PY{n}{data}\PY{p}{[}\PY{n}{data}\PY{p}{[}\PY{l+s+s1}{\PYZsq{}}\PY{l+s+s1}{text}\PY{l+s+s1}{\PYZsq{}}\PY{p}{]}\PY{o}{.}\PY{n}{str}\PY{o}{.}\PY{n}{contains}\PY{p}{(}\PY{l+s+s2}{\PYZdq{}}\PY{l+s+s2}{@}\PY{l+s+s2}{\PYZdq{}}\PY{p}{)} \PY{o}{==} \PY{k+kc}{True}\PY{p}{]}
         \PY{n}{mentions} \PY{o}{=}  \PY{n}{mentions}\PY{p}{[}\PY{n}{mentions}\PY{p}{[}\PY{l+s+s1}{\PYZsq{}}\PY{l+s+s1}{in\PYZus{}reply\PYZus{}to\PYZus{}screen\PYZus{}name}\PY{l+s+s1}{\PYZsq{}}\PY{p}{]}\PY{o}{.}\PY{n}{notnull}\PY{p}{(}\PY{p}{)} \PY{o}{==} \PY{k+kc}{False}\PY{p}{]}
         \PY{n}{mentions} \PY{o}{=} \PY{n}{mentions}\PY{p}{[}\PY{n}{mentions}\PY{p}{[}\PY{l+s+s1}{\PYZsq{}}\PY{l+s+s1}{text}\PY{l+s+s1}{\PYZsq{}}\PY{p}{]}\PY{o}{.}\PY{n}{str}\PY{o}{.}\PY{n}{startswith}\PY{p}{(}\PY{l+s+s2}{\PYZdq{}}\PY{l+s+s2}{RT}\PY{l+s+s2}{\PYZdq{}}\PY{p}{)} \PY{o}{==} \PY{k+kc}{False}\PY{p}{]}
         
         \PY{n}{formattedRetweets} \PY{o}{=} \PY{n}{plotter}\PY{o}{.}\PY{n}{retweetsToTwoUsers}\PY{p}{(}\PY{n}{retweets}\PY{p}{)}
         \PY{n}{formattedReplies} \PY{o}{=} \PY{n}{plotter}\PY{o}{.}\PY{n}{repliesToTwoUsers}\PY{p}{(}\PY{n}{replies}\PY{p}{)}
         \PY{n}{formattedMentions} \PY{o}{=} \PY{n}{plotter}\PY{o}{.}\PY{n}{mentionsToTwoUsers}\PY{p}{(}\PY{n}{mentions}\PY{p}{)}
         
         \PY{n}{interactions} \PY{o}{=} \PY{n}{np}\PY{o}{.}\PY{n}{concatenate}\PY{p}{(}\PY{p}{(}\PY{n}{formattedReplies}\PY{p}{,} \PY{n}{formattedMentions}\PY{p}{,} \PY{n}{formattedRetweets}\PY{p}{)}\PY{p}{,} \PY{n}{axis}\PY{o}{=}\PY{l+m+mi}{0}\PY{p}{)}
         
         \PY{c+c1}{\PYZsh{} above code previously explained in the \PYZdq{}Finding and analysing user interactions\PYZdq{} section}
         
         \PY{n}{plotter} \PY{o}{=} \PY{n}{Plotter}\PY{p}{(}\PY{p}{)}
         \PY{n}{plotter}\PY{o}{.}\PY{n}{networkGraph}\PY{p}{(}\PY{n}{interactions}\PY{p}{,} \PY{l+s+s2}{\PYZdq{}}\PY{l+s+s2}{interactions}\PY{l+s+s2}{\PYZdq{}}\PY{p}{)}
\end{Verbatim}


    \begin{Verbatim}[commandchars=\\\{\}]
Now drawing the graph showing all interactions between users.

    \end{Verbatim}

    \begin{center}
    \adjustimage{max size={0.9\linewidth}{0.9\paperheight}}{output_61_1.png}
    \end{center}
    { \hspace*{\fill} \\}
    
    \begin{Verbatim}[commandchars=\\\{\}]
Graph density = 2.5182130445852386e-05

    \end{Verbatim}

    The networkGraph() method

The networkgraph() method uses networkx, imported as nx. It first
performs a print statement informing the user of what the graph
represents, before generating and displaying the graph. The list of
username pairs in the DataFrame given are iterated over, and an edge is
added between each pair.The width of the lines are adjusted to best
display the connections between nodes for particular inputs (found by
trial and error). The parameters are then passed to generate and then
display the graph, before some properties of the graph, found with
library functions, are printed.

    \begin{Verbatim}[commandchars=\\\{\}]
{\color{incolor}In [{\color{incolor}31}]:} \PY{k}{def} \PY{n+nf}{networkGraph}\PY{p}{(}\PY{n+nb+bp}{self}\PY{p}{,} \PY{n}{plotItemList}\PY{p}{,} \PY{n}{whatPlotting}\PY{p}{)}\PY{p}{:}
             \PY{n}{G} \PY{o}{=} \PY{n}{nx}\PY{o}{.}\PY{n}{DiGraph}\PY{p}{(}\PY{p}{)}
         
             \PY{n+nb}{print}\PY{p}{(}\PY{l+s+s2}{\PYZdq{}}\PY{l+s+s2}{Now drawing the graph showing all }\PY{l+s+s2}{\PYZdq{}} \PY{o}{+} \PY{n}{whatPlotting} \PY{o}{+} \PY{l+s+s2}{\PYZdq{}}\PY{l+s+s2}{ between users.}\PY{l+s+s2}{\PYZdq{}}\PY{p}{)}
         
             \PY{k}{if} \PY{p}{(}\PY{n}{whatPlotting} \PY{o}{==} \PY{l+s+s2}{\PYZdq{}}\PY{l+s+s2}{interactions}\PY{l+s+s2}{\PYZdq{}}\PY{p}{)}\PY{p}{:} \PY{n}{plotItemList} \PY{o}{=} \PY{n}{DataFrame}\PY{p}{(}\PY{n}{plotItemList}\PY{p}{)}
         
             \PY{k}{for} \PY{n}{index}\PY{p}{,} \PY{n}{row} \PY{o+ow}{in} \PY{n}{plotItemList}\PY{o}{.}\PY{n}{iterrows}\PY{p}{(}\PY{p}{)}\PY{p}{:}
                 \PY{n}{G}\PY{o}{.}\PY{n}{add\PYZus{}edge}\PY{p}{(}\PY{n}{row}\PY{p}{[}\PY{l+m+mi}{0}\PY{p}{]}\PY{p}{,} \PY{n}{row}\PY{p}{[}\PY{l+m+mi}{1}\PY{p}{]}\PY{p}{)}
             \PY{n}{UG} \PY{o}{=} \PY{n}{G}\PY{o}{.}\PY{n}{to\PYZus{}undirected}\PY{p}{(}\PY{p}{)}
         
             \PY{n}{width} \PY{o}{=} \PY{l+m+mf}{0.003}
             \PY{k}{if} \PY{p}{(}\PY{n}{whatPlotting} \PY{o}{==} \PY{l+s+s2}{\PYZdq{}}\PY{l+s+s2}{replies}\PY{l+s+s2}{\PYZdq{}}\PY{p}{)}\PY{p}{:} \PY{n}{width} \PY{o}{=} \PY{l+m+mf}{0.4}
             \PY{k}{if} \PY{p}{(}\PY{n}{whatPlotting} \PY{o}{==} \PY{l+s+s2}{\PYZdq{}}\PY{l+s+s2}{mentions}\PY{l+s+s2}{\PYZdq{}}\PY{p}{)}\PY{p}{:} \PY{n}{width} \PY{o}{=} \PY{l+m+mf}{0.07}
         
             \PY{n}{options} \PY{o}{=} \PY{p}{\PYZob{}}
                 \PY{l+s+s1}{\PYZsq{}}\PY{l+s+s1}{node\PYZus{}color}\PY{l+s+s1}{\PYZsq{}}\PY{p}{:} \PY{l+s+s1}{\PYZsq{}}\PY{l+s+s1}{black}\PY{l+s+s1}{\PYZsq{}}\PY{p}{,}
                 \PY{l+s+s1}{\PYZsq{}}\PY{l+s+s1}{node\PYZus{}size}\PY{l+s+s1}{\PYZsq{}}\PY{p}{:} \PY{l+m+mi}{1}\PY{p}{,}
                 \PY{l+s+s1}{\PYZsq{}}\PY{l+s+s1}{line\PYZus{}color}\PY{l+s+s1}{\PYZsq{}}\PY{p}{:} \PY{l+s+s1}{\PYZsq{}}\PY{l+s+s1}{grey}\PY{l+s+s1}{\PYZsq{}}\PY{p}{,}
                 \PY{l+s+s1}{\PYZsq{}}\PY{l+s+s1}{linewidths}\PY{l+s+s1}{\PYZsq{}}\PY{p}{:} \PY{l+m+mi}{0}\PY{p}{,}
                 \PY{l+s+s1}{\PYZsq{}}\PY{l+s+s1}{width}\PY{l+s+s1}{\PYZsq{}}\PY{p}{:} \PY{n}{width}\PY{p}{,}
             \PY{p}{\PYZcb{}}
         
             \PY{n}{nx}\PY{o}{.}\PY{n}{draw\PYZus{}circular}\PY{p}{(}\PY{n}{UG}\PY{p}{,} \PY{o}{*}\PY{o}{*}\PY{n}{options}\PY{p}{)}
             \PY{n}{plt}\PY{o}{.}\PY{n}{show}\PY{p}{(}\PY{p}{)}
             \PY{n+nb}{print}\PY{p}{(}\PY{l+s+s2}{\PYZdq{}}\PY{l+s+s2}{Graph density = }\PY{l+s+s2}{\PYZdq{}} \PY{o}{+} \PY{n+nb}{str}\PY{p}{(}\PY{n}{nx}\PY{o}{.}\PY{n}{density}\PY{p}{(}\PY{n}{G}\PY{p}{)}\PY{p}{)}\PY{p}{)} \PY{c+c1}{\PYZsh{} TODO: print some more properties of the graph}
             \PY{k}{if} \PY{p}{(}\PY{n}{whatPlotting} \PY{o}{!=} \PY{l+s+s2}{\PYZdq{}}\PY{l+s+s2}{interactions}\PY{l+s+s2}{\PYZdq{}} \PY{o+ow}{and} \PY{n}{whatPlotting} \PY{o}{!=} \PY{l+s+s2}{\PYZdq{}}\PY{l+s+s2}{mentions}\PY{l+s+s2}{\PYZdq{}}\PY{p}{)}\PY{p}{:} \PY{n+nb}{print}\PY{p}{(}\PY{l+s+s2}{\PYZdq{}}\PY{l+s+se}{\PYZbs{}n}\PY{l+s+se}{\PYZbs{}n}\PY{l+s+s2}{\PYZdq{}}\PY{p}{)}
\end{Verbatim}


    Network Graphs of Retweets, Replies, and Mentions

    \begin{Verbatim}[commandchars=\\\{\}]
{\color{incolor}In [{\color{incolor}32}]:} \PY{k+kn}{from} \PY{n+nn}{plotter} \PY{k}{import} \PY{o}{*}
         \PY{k+kn}{from} \PY{n+nn}{parsing} \PY{k}{import} \PY{o}{*}
         
         \PY{n}{plotter} \PY{o}{=} \PY{n}{Plotter}\PY{p}{(}\PY{p}{)}
         
         \PY{n}{data} \PY{o}{=} \PY{n}{CensusReader}\PY{p}{(}\PY{l+s+s2}{\PYZdq{}}\PY{l+s+s2}{data/CometLanding\PYZus{}refined.csv}\PY{l+s+s2}{\PYZdq{}}\PY{p}{)}\PY{o}{.}\PY{n}{data}
         
         \PY{n}{replies} \PY{o}{=} \PY{n}{data}\PY{p}{[}\PY{n}{data}\PY{p}{[}\PY{l+s+s1}{\PYZsq{}}\PY{l+s+s1}{in\PYZus{}reply\PYZus{}to\PYZus{}screen\PYZus{}name}\PY{l+s+s1}{\PYZsq{}}\PY{p}{]}\PY{o}{.}\PY{n}{notnull}\PY{p}{(}\PY{p}{)}\PY{p}{]}
         \PY{n}{retweets} \PY{o}{=} \PY{n}{data}\PY{p}{[}\PY{n}{data}\PY{p}{[}\PY{l+s+s1}{\PYZsq{}}\PY{l+s+s1}{text}\PY{l+s+s1}{\PYZsq{}}\PY{p}{]}\PY{o}{.}\PY{n}{str}\PY{o}{.}\PY{n}{startswith}\PY{p}{(}\PY{l+s+s2}{\PYZdq{}}\PY{l+s+s2}{RT}\PY{l+s+s2}{\PYZdq{}}\PY{p}{)} \PY{o}{==} \PY{k+kc}{True}\PY{p}{]}
         
         \PY{n}{mentions} \PY{o}{=} \PY{n}{data}\PY{p}{[}\PY{n}{data}\PY{p}{[}\PY{l+s+s1}{\PYZsq{}}\PY{l+s+s1}{text}\PY{l+s+s1}{\PYZsq{}}\PY{p}{]}\PY{o}{.}\PY{n}{str}\PY{o}{.}\PY{n}{contains}\PY{p}{(}\PY{l+s+s2}{\PYZdq{}}\PY{l+s+s2}{@}\PY{l+s+s2}{\PYZdq{}}\PY{p}{)} \PY{o}{==} \PY{k+kc}{True}\PY{p}{]}
         \PY{n}{mentions} \PY{o}{=}  \PY{n}{mentions}\PY{p}{[}\PY{n}{mentions}\PY{p}{[}\PY{l+s+s1}{\PYZsq{}}\PY{l+s+s1}{in\PYZus{}reply\PYZus{}to\PYZus{}screen\PYZus{}name}\PY{l+s+s1}{\PYZsq{}}\PY{p}{]}\PY{o}{.}\PY{n}{notnull}\PY{p}{(}\PY{p}{)} \PY{o}{==} \PY{k+kc}{False}\PY{p}{]}
         \PY{n}{mentions} \PY{o}{=} \PY{n}{mentions}\PY{p}{[}\PY{n}{mentions}\PY{p}{[}\PY{l+s+s1}{\PYZsq{}}\PY{l+s+s1}{text}\PY{l+s+s1}{\PYZsq{}}\PY{p}{]}\PY{o}{.}\PY{n}{str}\PY{o}{.}\PY{n}{startswith}\PY{p}{(}\PY{l+s+s2}{\PYZdq{}}\PY{l+s+s2}{RT}\PY{l+s+s2}{\PYZdq{}}\PY{p}{)} \PY{o}{==} \PY{k+kc}{False}\PY{p}{]}
         
         \PY{n}{formattedRetweets} \PY{o}{=} \PY{n}{plotter}\PY{o}{.}\PY{n}{retweetsToTwoUsers}\PY{p}{(}\PY{n}{retweets}\PY{p}{)}
         \PY{n}{formattedReplies} \PY{o}{=} \PY{n}{plotter}\PY{o}{.}\PY{n}{repliesToTwoUsers}\PY{p}{(}\PY{n}{replies}\PY{p}{)}
         \PY{n}{formattedMentions} \PY{o}{=} \PY{n}{plotter}\PY{o}{.}\PY{n}{mentionsToTwoUsers}\PY{p}{(}\PY{n}{mentions}\PY{p}{)}
         
         \PY{c+c1}{\PYZsh{} for code explanation see \PYZdq{}Network graph of all interactions\PYZdq{} section}
         
         \PY{n}{plotter} \PY{o}{=} \PY{n}{Plotter}\PY{p}{(}\PY{p}{)}
         \PY{n}{plotter}\PY{o}{.}\PY{n}{networkGraph}\PY{p}{(}\PY{n}{formattedRetweets}\PY{p}{,} \PY{l+s+s2}{\PYZdq{}}\PY{l+s+s2}{retweets}\PY{l+s+s2}{\PYZdq{}}\PY{p}{)}
         \PY{n}{plotter}\PY{o}{.}\PY{n}{networkGraph}\PY{p}{(}\PY{n}{formattedReplies}\PY{p}{,} \PY{l+s+s2}{\PYZdq{}}\PY{l+s+s2}{replies}\PY{l+s+s2}{\PYZdq{}}\PY{p}{)}
         \PY{n}{plotter}\PY{o}{.}\PY{n}{networkGraph}\PY{p}{(}\PY{n}{formattedMentions}\PY{p}{,} \PY{l+s+s2}{\PYZdq{}}\PY{l+s+s2}{mentions}\PY{l+s+s2}{\PYZdq{}}\PY{p}{)}
\end{Verbatim}


    \begin{Verbatim}[commandchars=\\\{\}]
Now drawing the graph showing all retweets between users.

    \end{Verbatim}

    \begin{center}
    \adjustimage{max size={0.9\linewidth}{0.9\paperheight}}{output_65_1.png}
    \end{center}
    { \hspace*{\fill} \\}
    
    \begin{Verbatim}[commandchars=\\\{\}]
Graph density = 2.692865021425508e-05



Now drawing the graph showing all replies between users.

    \end{Verbatim}

    \begin{center}
    \adjustimage{max size={0.9\linewidth}{0.9\paperheight}}{output_65_3.png}
    \end{center}
    { \hspace*{\fill} \\}
    
    \begin{Verbatim}[commandchars=\\\{\}]
Graph density = 0.000364948893811214



Now drawing the graph showing all mentions between users.

    \end{Verbatim}

    \begin{center}
    \adjustimage{max size={0.9\linewidth}{0.9\paperheight}}{output_65_5.png}
    \end{center}
    { \hspace*{\fill} \\}
    
    \begin{Verbatim}[commandchars=\\\{\}]
Graph density = 0.00021895905282390816

    \end{Verbatim}

    Pie Chart Showing Make-up of Interactions By Interaction Type

    \begin{Verbatim}[commandchars=\\\{\}]
{\color{incolor}In [{\color{incolor}33}]:} \PY{k+kn}{from} \PY{n+nn}{plotter} \PY{k}{import} \PY{o}{*}
         \PY{k+kn}{from} \PY{n+nn}{parsing} \PY{k}{import} \PY{o}{*}
         
         \PY{n}{plotter} \PY{o}{=} \PY{n}{Plotter}\PY{p}{(}\PY{p}{)}
         
         \PY{n}{data} \PY{o}{=} \PY{n}{CensusReader}\PY{p}{(}\PY{l+s+s2}{\PYZdq{}}\PY{l+s+s2}{data/CometLanding\PYZus{}refined.csv}\PY{l+s+s2}{\PYZdq{}}\PY{p}{)}\PY{o}{.}\PY{n}{data}
         
         \PY{n}{replies} \PY{o}{=} \PY{n}{data}\PY{p}{[}\PY{n}{data}\PY{p}{[}\PY{l+s+s1}{\PYZsq{}}\PY{l+s+s1}{in\PYZus{}reply\PYZus{}to\PYZus{}screen\PYZus{}name}\PY{l+s+s1}{\PYZsq{}}\PY{p}{]}\PY{o}{.}\PY{n}{notnull}\PY{p}{(}\PY{p}{)}\PY{p}{]}
         \PY{n}{retweets} \PY{o}{=} \PY{n}{data}\PY{p}{[}\PY{n}{data}\PY{p}{[}\PY{l+s+s1}{\PYZsq{}}\PY{l+s+s1}{text}\PY{l+s+s1}{\PYZsq{}}\PY{p}{]}\PY{o}{.}\PY{n}{str}\PY{o}{.}\PY{n}{startswith}\PY{p}{(}\PY{l+s+s2}{\PYZdq{}}\PY{l+s+s2}{RT}\PY{l+s+s2}{\PYZdq{}}\PY{p}{)} \PY{o}{==} \PY{k+kc}{True}\PY{p}{]}
         
         \PY{n}{mentions} \PY{o}{=} \PY{n}{data}\PY{p}{[}\PY{n}{data}\PY{p}{[}\PY{l+s+s1}{\PYZsq{}}\PY{l+s+s1}{text}\PY{l+s+s1}{\PYZsq{}}\PY{p}{]}\PY{o}{.}\PY{n}{str}\PY{o}{.}\PY{n}{contains}\PY{p}{(}\PY{l+s+s2}{\PYZdq{}}\PY{l+s+s2}{@}\PY{l+s+s2}{\PYZdq{}}\PY{p}{)} \PY{o}{==} \PY{k+kc}{True}\PY{p}{]}
         \PY{n}{mentions} \PY{o}{=}  \PY{n}{mentions}\PY{p}{[}\PY{n}{mentions}\PY{p}{[}\PY{l+s+s1}{\PYZsq{}}\PY{l+s+s1}{in\PYZus{}reply\PYZus{}to\PYZus{}screen\PYZus{}name}\PY{l+s+s1}{\PYZsq{}}\PY{p}{]}\PY{o}{.}\PY{n}{notnull}\PY{p}{(}\PY{p}{)} \PY{o}{==} \PY{k+kc}{False}\PY{p}{]}
         \PY{n}{mentions} \PY{o}{=} \PY{n}{mentions}\PY{p}{[}\PY{n}{mentions}\PY{p}{[}\PY{l+s+s1}{\PYZsq{}}\PY{l+s+s1}{text}\PY{l+s+s1}{\PYZsq{}}\PY{p}{]}\PY{o}{.}\PY{n}{str}\PY{o}{.}\PY{n}{startswith}\PY{p}{(}\PY{l+s+s2}{\PYZdq{}}\PY{l+s+s2}{RT}\PY{l+s+s2}{\PYZdq{}}\PY{p}{)} \PY{o}{==} \PY{k+kc}{False}\PY{p}{]}
         
         \PY{n}{formattedRetweets} \PY{o}{=} \PY{n}{plotter}\PY{o}{.}\PY{n}{retweetsToTwoUsers}\PY{p}{(}\PY{n}{retweets}\PY{p}{)}
         \PY{n}{formattedReplies} \PY{o}{=} \PY{n}{plotter}\PY{o}{.}\PY{n}{repliesToTwoUsers}\PY{p}{(}\PY{n}{replies}\PY{p}{)}
         \PY{n}{formattedMentions} \PY{o}{=} \PY{n}{plotter}\PY{o}{.}\PY{n}{mentionsToTwoUsers}\PY{p}{(}\PY{n}{mentions}\PY{p}{)}
         
         \PY{c+c1}{\PYZsh{} for explanation of above code, see \PYZdq{}Finding and analysing user interactions\PYZdq{} section}
         
         \PY{n}{plotter} \PY{o}{=} \PY{n}{Plotter}\PY{p}{(}\PY{p}{)}
         \PY{n}{plotter}\PY{o}{.}\PY{n}{interactionsPieChart}\PY{p}{(}\PY{n+nb}{len}\PY{p}{(}\PY{n}{formattedRetweets}\PY{p}{)}\PY{p}{,} \PY{n+nb}{len}\PY{p}{(}\PY{n}{formattedReplies}\PY{p}{)}\PY{p}{,} \PY{n+nb}{len}\PY{p}{(}\PY{n}{formattedMentions}\PY{p}{)}\PY{p}{)}
\end{Verbatim}


    \begin{center}
    \adjustimage{max size={0.9\linewidth}{0.9\paperheight}}{output_67_0.png}
    \end{center}
    { \hspace*{\fill} \\}
    
    interactionsPieChart() method

The interactionsPieChart() method takes in the number of retweets,
replies, and mentions. It then calculates the percentage shares of each
interaction type, out of all interactions. It then generates the labels
for each type of interaction that include its percentage share of all
interactions. These arguments are then passed into various functions
from plt (matplotlib.pyplot) to generate the pie chart, before it is
given a title and then displayed.

    \begin{Verbatim}[commandchars=\\\{\}]
{\color{incolor}In [{\color{incolor}34}]:} \PY{k}{def} \PY{n+nf}{interactionsPieChart}\PY{p}{(}\PY{n+nb+bp}{self}\PY{p}{,} \PY{n}{retweets}\PY{p}{,} \PY{n}{replies}\PY{p}{,} \PY{n}{mentions}\PY{p}{)}\PY{p}{:}
             \PY{n}{total} \PY{o}{=} \PY{n}{retweets} \PY{o}{+} \PY{n}{replies} \PY{o}{+} \PY{n}{mentions}
             \PY{c+c1}{\PYZsh{} The slices will be ordered and plotted counter\PYZhy{}clockwise.}
             \PY{n}{percents} \PY{o}{=} \PY{p}{[}\PY{p}{(}\PY{n+nb}{float}\PY{p}{(}\PY{n}{retweets}\PY{p}{)}\PY{o}{/}\PY{n+nb}{float}\PY{p}{(}\PY{n}{total}\PY{p}{)}\PY{o}{*}\PY{l+m+mf}{100.0}\PY{p}{)}\PY{p}{,} \PY{p}{(}\PY{n+nb}{float}\PY{p}{(}\PY{n}{replies}\PY{p}{)}\PY{o}{/}\PY{n+nb}{float}\PY{p}{(}\PY{n}{total}\PY{p}{)}\PY{o}{*}\PY{l+m+mf}{100.0}\PY{p}{)}\PY{p}{,}\PY{p}{(}\PY{n+nb}{float}\PY{p}{(}\PY{n}{mentions}\PY{p}{)}\PY{o}{/}\PY{n+nb}{float}\PY{p}{(}\PY{n}{total}\PY{p}{)}\PY{o}{*}\PY{l+m+mf}{100.0}\PY{p}{)}\PY{p}{]}
             
             \PY{n}{labels} \PY{o}{=} \PY{l+s+s1}{\PYZsq{}}\PY{l+s+s1}{Retweets (}\PY{l+s+s1}{\PYZsq{}} \PY{o}{+} \PY{n+nb}{str}\PY{p}{(}\PY{n}{percents}\PY{p}{[}\PY{l+m+mi}{0}\PY{p}{]}\PY{p}{)} \PY{o}{+} \PY{l+s+s1}{\PYZsq{}}\PY{l+s+s1}{\PYZpc{}}\PY{l+s+s1}{)}\PY{l+s+s1}{\PYZsq{}}\PY{p}{,} \PY{l+s+s1}{\PYZsq{}}\PY{l+s+s1}{Replies (}\PY{l+s+s1}{\PYZsq{}} \PY{o}{+} \PY{n+nb}{str}\PY{p}{(}\PY{n}{percents}\PY{p}{[}\PY{l+m+mi}{1}\PY{p}{]}\PY{p}{)} \PY{o}{+} \PY{l+s+s1}{\PYZsq{}}\PY{l+s+s1}{\PYZpc{}}\PY{l+s+s1}{)}\PY{l+s+s1}{\PYZsq{}}\PY{p}{,} \PY{l+s+s1}{\PYZsq{}}\PY{l+s+s1}{Mentions (}\PY{l+s+s1}{\PYZsq{}} \PY{o}{+} \PY{n+nb}{str}\PY{p}{(}\PY{n}{percents}\PY{p}{[}\PY{l+m+mi}{2}\PY{p}{]}\PY{p}{)} \PY{o}{+} \PY{l+s+s1}{\PYZsq{}}\PY{l+s+s1}{\PYZpc{}}\PY{l+s+s1}{)}\PY{l+s+s1}{\PYZsq{}}
             \PY{n}{fracs} \PY{o}{=} \PY{p}{[}\PY{n}{retweets}\PY{p}{,} \PY{n}{replies}\PY{p}{,} \PY{n}{mentions}\PY{p}{]}
             \PY{n}{colors} \PY{o}{=} \PY{p}{[}\PY{l+s+s1}{\PYZsq{}}\PY{l+s+s1}{\PYZsh{}ff4d4d}\PY{l+s+s1}{\PYZsq{}}\PY{p}{,} \PY{l+s+s1}{\PYZsq{}}\PY{l+s+s1}{\PYZsh{}0FAC36}\PY{l+s+s1}{\PYZsq{}}\PY{p}{,} \PY{l+s+s1}{\PYZsq{}}\PY{l+s+s1}{\PYZsh{}0F18AC}\PY{l+s+s1}{\PYZsq{}}\PY{p}{]} \PY{c+c1}{\PYZsh{} red, green, blue}
         
             \PY{n}{patches}\PY{p}{,} \PY{n}{texts} \PY{o}{=} \PY{n}{plt}\PY{o}{.}\PY{n}{pie}\PY{p}{(}\PY{n}{fracs}\PY{p}{,} \PY{n}{colors}\PY{o}{=}\PY{n}{colors}\PY{p}{,} \PY{n}{shadow}\PY{o}{=}\PY{k+kc}{True}\PY{p}{,} \PY{n}{startangle}\PY{o}{=}\PY{l+m+mi}{90}\PY{p}{)}
             \PY{n}{plt}\PY{o}{.}\PY{n}{legend}\PY{p}{(}\PY{n}{patches}\PY{p}{,} \PY{n}{labels}\PY{p}{,} \PY{n}{loc}\PY{o}{=}\PY{l+s+s2}{\PYZdq{}}\PY{l+s+s2}{best}\PY{l+s+s2}{\PYZdq{}}\PY{p}{)}
             \PY{n}{plt}\PY{o}{.}\PY{n}{axis}\PY{p}{(}\PY{l+s+s1}{\PYZsq{}}\PY{l+s+s1}{equal}\PY{l+s+s1}{\PYZsq{}}\PY{p}{)}
         
             \PY{n}{plt}\PY{o}{.}\PY{n}{tight\PYZus{}layout}\PY{p}{(}\PY{p}{)}
             \PY{n}{plt}\PY{o}{.}\PY{n}{title}\PY{p}{(}\PY{l+s+s1}{\PYZsq{}}\PY{l+s+s1}{Pie Chart Showing Frequencies of Different Interaction Types}\PY{l+s+s1}{\PYZsq{}}\PY{p}{)}
             \PY{n}{plt}\PY{o}{.}\PY{n}{show}\PY{p}{(}\PY{p}{)}
\end{Verbatim}



    % Add a bibliography block to the postdoc
    
    
    
    \end{document}
